\documentclass[draft]{article}

\usepackage{amssymb}
\usepackage{amstext}
\usepackage{amsmath}
\usepackage{graphicx}
\usepackage[colorlinks=true, urlcolor=blue, pdfborder={0 0 0}]{hyperref}
\usepackage{verbatim}

\tolerance=500
\newcommand*{\urldecorated}[1]{[\url{#1}]}

\newlength{\firstcolwidth}

\newcommand*{\secondcolindent}[3][2em]{
\par
{
\parindent=#1
\settowidth{\firstcolwidth}{#2}
\addtolength{\firstcolwidth}{#1}
\setlength{\hangindent}{\firstcolwidth}
#2#3
\par
}
}

\newcommand*{\ProgramInvokation}[2]{
\secondcolindent[2em]{\texttt{java -cp ape.jar \mbox{ru.autosome.#1} }}{\texttt{#2}}
}

\newcommand*{\DefineProgramInvokationCommand}[2]{
\newcommand{#1}[1]{
\ProgramInvokation{#2}{##1}
}
}

\DefineProgramInvokationCommand{\EvalSimilarity}{macroape.EvalSimilarity}
\DefineProgramInvokationCommand{\ScanCollection}{macroape.ScanCollection}
\DefineProgramInvokationCommand{\FindThreshold}{ape.FindThreshold}
\DefineProgramInvokationCommand{\FindPvalue}{ape.FindPvalue}
\DefineProgramInvokationCommand{\PrecalculateThresholds}{ape.PrecalculateThresholds}
\DefineProgramInvokationCommand{\SNPScan}{perfectosape.SNPScan}


\newcommand*{\example}[1]{\par\vspace{2ex}\noindent\textbf{Example#1:}\par\nopagebreak\vspace{0.5ex plus 0.3ex minus 0.1ex}}
\newcommand*{\exampleof}[1]{\example{ (#1)}}
\newcommand*{\usageheader}[1][Usage]{\par\noindent\textbf{#1:}\par\nopagebreak}
\newcommand*{\outputheader}[1][Output]{\par\noindent\textbf{#1:}\par\nopagebreak}
\newcommand*{\cmdoutput}[1]{{\small#1}}
\newcommand*{\cmdoutputfromfile}[1]{\cmdoutput{\verbatiminput{./ProgramOutput/#1}}}


\newcommand*{\inangles}[1]{\textless#1\textgreater}
\newcommand*{\requiredarg}[1]{\inangles{#1}}
\newcommand*{\optionalarg}[1]{[#1]}
\newcommand*{\cmdname}[1]{\mbox{\texttt{#1}}}
\newcommand*{\cmdoption}[1]{\mbox{\texttt{#1}}}

\newcommand*{\pvalue}{\mbox{P-value}}
\newcommand*{\pvalues}{\mbox{P-values}}

\begin{document}

\title{
MACRO-PERFECTOS-APE~---\\{\small MAtrix CompaRisOn \&\\ Predicting Regulatory Functional Effect of SNPs\\ by Approximate \pvalue\ Estimation}\\
-- User Manual --
}
\maketitle

\section{Abstract}
Here we present MACRO-APE and PERFECTOS-APE software designed for practical sequence analysis involving classic mononucleotide and dinucleotide position weight matrices (PWMs) of DNA sequence patterns often called \textit{motifs}. The common usage case for DNA motifs is representation of transcription factor binding sites.

The software allows (1) comparing different PWMs using a variant of Jaccard similarity measure, e.g. scanning a motif collection for motifs similar to a given query, (2) analysing single-nucleotide variants for possible regulatory effect through transcription factor affinity changes, (3) performing basic PWM analysis (\pvalue\ and threshold estimation).


MACRO- and PERFECTOS-APE require Java Runtime Environment 1.6 (or newer) to run.
Thus *-APE should be able to function under most modern operating systems.

Several existing motif collections such as HOCOMOCO as well as several individual PWM examples are available to be used with the *-APE package:
HOCOMOCO \urldecorated{http://autosome.ru/HOCOMOCO/} TFBS model collection and several examples of PWMs (motifs) can be downloaded with MACRO-PERFECTOS-APE at \urldecorated{http://opera.autosome.ru/downloads/all_collections_pack.tar.gz}.

Windows users can get the latest Java directly from Oracle: \urldecorated{http://www.java.com}.
Modern Linux distributions typically have OpenJDK preinstalled, otherwise it should be available via a distribution-specific package manager.

The latest MACRO-PERFECTOS-APE package can be found at \urldecorated{http://opera.autosome.ru/downloads/ape.jar}. Source codes are distributed under WTFPL public license. They are available in a github repository: \urldecorated{https://github.com/prijutme4ty/macro-perfectos-ape} and as a single archive at \urldecorated{http://opera.autosome.ru/downloads/macro-perfectos-ape_src.jar}.

Web version (only basic functionality available) can be found at \urldecorated{http://opera.autosome.ru}.

This manual is also hosted on github in a repository: \urldecorated{https://github.com/prijutme4ty/macro-perfectos-ape-manual}.


\section{Overview}
All tools are packed in a jar-file with compiled Java classes. There are three principal packages for tools:
\texttt{ru.autosome.ape}, \texttt{ru.autosome.macroape} and \texttt{ru.autosome.perfectosape}.

\textbf{APE} in \texttt{ru.autosome.ape} stands for Approximate \pvalue\ Estimation, this package contains basic tools:
\begin{itemize}
\item \cmdname{FindThreshold}~--- to find PWM threshold for a given \pvalue
\item \cmdname{FindPvalue}~--- to find PWM \pvalue\ corresponding to a given threshold
\item \cmdname{PrecalculateThresholds}~--- to precalculate lists of thresholds by \pvalues\ for a motif collection
\end{itemize}

\textbf{MACRO-APE} in \texttt{ru.autosome.macroape} denotes MAtrix CompaRisOn by Approximate \pvalue\ Estimation. Package consists of several tools related to motif comparison:
\begin{itemize}
\item \cmdname{EvalSimilarity}~--- to evaluate similarity between a pair of PWMs.
\item \cmdname{ScanCollection}~--- to search a collection of motifs for PWMs similar to a given one.
\item \cmdname{CollectDistanceMatrix}~--- to collect pairwise distance matrix for a motif collection which can be further used for motif clustering. (Not yet finalized tool!)
\item \cmdname{AlignMotifs}~--- to evaluate the best motif relative alignment. (This tool not yet moved from ruby to Java!)
\end{itemize}

\textbf{PERFECTOS-APE} in \texttt{ru.autosome.perfectosape} denotes Predicting Regulatory Functional Effect of SNPs by Approximate \pvalue\ Estimation. Package consists of a the only tool:
\begin{itemize}
\item\cmdname{SNPScan}~--- to search a pack of sequences with SNVs against a collection of PWMs for (SNV, PWM) pairs, such that single nucleotide substitution induces significant affinity change of a given PWM to a given sequence.
\end{itemize}

All tools use similar command-line format. The examples are shown under the assumtion that the *-APE package \texttt{ape.jar} is located in the current folder (working directory). A typical command line will look like:\par
\texttt{\secondcolindent{java -cp ape.jar }{ru.autosome.ToolName\\ \requiredarg{required arguments}\\ \optionalarg{options}}}

Each tool can be used with \cmdoption{--help} or \cmdoption{-h} options to display a detailed help message describing order of arguments and a list of optional parameters.

Each tool is provided in mononucleotide and dinucleotide versions for mono- and diPWMs and respective background models.
Generally, mononucleotide version has wider application range, since most of existing motif collections provide only basic mononucleotide PWMs.
Naming convention is the same for all tools: mononucleotide version is located in package's root, dinucleotide version has the same name but is located in a subpackage \texttt{".di"}.

E.g. for \cmdname{ape.FindThreshold} the full class names are:
\begin{itemize}
\item\cmdname{ru.autosome.ape.FindThreshold} for mononucleotide version
\item\cmdname{ru.autosome.ape.di.FindThreshold} for dinucleotide version.
\end{itemize}

Please note, that dinucleotide tools use special input formats for dinucleotide Position Weight Matrices (diPWM) and respective background models. Input data formats are described in a special section.


\subsubsection{Output formats}

All tools except \cmdname{PrecalculateThresholds} print their results into the standard output stream (stdout).
\cmdname{PrecalculateThresholds} stores its results in a set of output files created in a specified folder.

For each tool the output can be redirected to a file using OS syntax, e.g. with a "\texttt{\textgreater}"-sign. For example:
\texttt{\FindPvalue{motifs/KLF4\_f2.pwm 3.3 5.0 7.1 \textgreater\ KLF4\_P-values.txt}}

Output generally consists of two types of lines. Lines starting with \texttt{"\#"} character (comments) show
input parameters and descriptions. The results are presented in non-commented lines.


\section{Basic APE tools}
APE tools are designed to properly convert PWM thresholds to \pvalues\ and vice versa.

Position weight matrix~(PWM) of DNA motifs assigns a score to each "word" (nucleotide sequence of a fixed length $l$). It makes 
possible to range the words by their scores, e.g. corresponding to predicted transcription affinity for PWMs of transcription factor binding sites~(TFBS). Given a threshold, one can divide all $l$-mers into two subsets: words whose score are not less than the threshold and the rest. Typically, the words passing the score threshold are selected for downstream analysis, e.g. they are considered as putative transcription factor binding sites. 

What is important, the threshold values are not directly comparable for different PWMs. One strategy to have a unified scale is to use motif \pvalues\ instead.

The \pvalue\ of a certain PWM and a score threshold is the probability to generate a word with the score not less than the threshold at random.

Inverse task is to estimate a threshold for a predefined \pvalue. In particular this allows to select a PWM score threshold corresponding to a predefined positive prediction rate across the $l$-mer dictionary (e.g. only $x\%$ of words are predicted as putative TFBS).

Our tools perform threshold -- \pvalue\ conversion implementing a dynamic programming algorithm on a granulated (discretized) PWM models using a simplified approach comparing to that described in [Touzet].

More details on \pvalues, thresholds and the algorithm are provided in the MACRO-APE paper. [MACROAPE] \urldecorated{http://www.almob.org/content/8/1/23}


\subsection{FindThreshold}
This is a stand-alone tool to search for a score threshold corresponding to a given \pvalue\ for a given PWM.
\cmdname{FindThreshold} requires a PWM and a \pvalue\ as input and returns a threshold for which the set of
words scoring with this PWM no less than the given threshold has the aggregated probability equal to
the given \pvalue. The program can process a set of \pvalues, and return a set of thresholds.
This tool implements a simplified algorithm derived from that implemented in the TFM-Pvalue software of
Helen Touzet \urldecorated{http://bioinfo.lifl.fr/TFM/TFMpvalue/} but with the fixed predefined discretization level (see "PWM
discretization" section below).

\usageheader
\FindThreshold{\requiredarg{motif file}\\ \optionalarg{list of \pvalues}}

\exampleof{motif file \texttt{KLF4\_f2.pat}, \pvalue\ of 0.001 and 0.0005}
\FindThreshold{motifs/KLF4\_f2.pat 0.001 0.0005}
% \outputheader
% \cmdoutputfromfile{FindThresholdOutput.txt}

NOTE! By default \cmdname{FindThreshold} looks for threshold large enough to obtain \pvalue\ not 
greater than requested (lower boundary for \pvalue). For details see \cmdoption{--boundary} option description.

\subsection{FindPvalue}
\cmdname{FindPvalue} is a stand-alone tool to find the \pvalue\ corresponding to a given threshold level for a given PWM.

\usageheader
\FindPvalue{\requiredarg{motif file}\\ \requiredarg{list of thresholds}}

\exampleof{motif file \texttt{KLF4\_f2.pat}, thresholds of 4.1719 and 5.2403}
\FindPvalue{motifs/KLF4\_f2.pat 4.1719 5.2403}
% \outputheader
% \cmdoutputfromfile{FindPvalueOutput.txt}

\subsection{PrecalculateThresholds}
This tool is intended to process the motif collection (a folder containing separate
files for each motif) and to store precomputed score distributions of motif PWMs. Each score distributions is saved as a sorted list of \mbox{(threshold,\pvalue)} pairs with \pvalues\ taken at uniform intervals at quantiles of score distribution. It allows for faster score -- \pvalue\ conversion performing binary search through a list of thresholds or \pvalues. \cmdname{PrecalculateThresholds} doesn't store precise score distribution because for a non-disretized PWM it can be extremely large with unpractical precision. Practically it's sufficient to estimate \pvalue\ with a specified error level of e.g. 5\%.

In order to use precalculated distribution several *-APE tools have \cmdoption{--precalc} option which takes a folder containing results of \cmdname{PrecalculateThresholds}.

\textbf{Note:} Precalculation allows notably increase speed of threshold to \pvalue\ calculation (up to 100x). Unfortunately it deals with a file system to load the precalculated data. Thus it's recomended to use precalculated score distribution for tasks where the same motif \pvalue\ evaluation is performed multiple times so that the score distribution is loaded once and used multiple times. At a moment the only use case is ~-- \cmdname{perfectosape.SNPScan} which assesses each of multiple SNPs against the same motif collection.
% TODO: yet undocumented CollectDistanceMatrix also can take an advantage of PrecalculateThresholds (but there are more effective approaches).

\textbf{Note:} Resulting score distribution depends on a discretization level and on a specified background model. It is up to the user to control that a score distribution was precomputed with the proper parameters. The parameters values are not anyhow stored after score distribution precalculation and are not implicitly contolled when reusing precomputed data.

\usageheader
\PrecalculateThresholds{\requiredarg{motif collection folder}\\ \requiredarg{output folder}\\ \optionalarg{options}}

\example{}
\PrecalculateThresholds{./motifs/ ./motif\_thresholds/}

This will create \texttt{./motif\_thresholds/} folder (if not already exist) and multiple files inside, one file per motif in \texttt{./motifs/} folder. For a given motif output file will be named as \texttt{\inangles{name of motif}.thr}.
Each file contains lines in the following format:

\texttt{\inangles{threshold} \textit{tab} \inangles{corresponding \pvalue}}

Lines are sorted with thresholds ascending (\pvalue\ descending).

It takes about half a minute to preprocess the collection of $\sim$400 mononucleotide PWMs with default
parameters using 1.5 GHz CPU.

To alter granularity of resulting \pvalues\ list one can use \cmdoption{--pvalues} option in the following format:\\
\cmdoption{--pvalues \requiredarg{from,to,step,mode}}\\

Parameters set the \pvalues\ progression in the resulting list. \pvalues\ can use arithmetic or geometric progession which corresponds to \texttt{add} or \texttt{mul} value of \texttt{mode}.

\texttt{from} and \texttt{to} represent progression boundaries and \cmdoption{step} corresponds to a common difference (\texttt{add}) or a common ratio (\texttt{mul}) of progression. Parameters are comma-separated without spaces between.

For example, default progression can be written as follows:\\
\cmdoption{--pvalues 1.0,1e-15,1.05,mul}\\

It means that \cmdname{PrecalculateThresholds} collect thresholds for
each of these \pvalues: $1.0, 1.0/1.05, 1.0/1.05^2, 1.0/1.05^3,\ldots,
10^{-15}$

To specify relative error of $\epsilon$ use geometric progression with common ratio of $1+\epsilon$ and boundaries: from $1.0$ to a minimal expected non-zero \pvalue.



\section{MACRO-APE: Matrix Comparison by Approximate P-value Estimation}
Let us have two PWMs with given threshold levels. The similarity between PWMs is
related to the number of words recognized by both PWMs
(or the aggregated probability of the word set under the given i.i.d. model).
To calculate this value we use generalized approach
described in [Touzet2007] for two PWMs simultaneously in a way similar to that in [Pape2008].
The number of words recognized by both PWMs can be used to construct a variant of Jaccard
similarity measure for motifs considered as sets of allowed words scoring no less than predefined thresholds.

Typical methods of PWM comparison are based on direct evaluation of matrix elements, for instance
by comparing matrices column by column (where different columns correspond to different 
positions of a transcription factor binding site).
On the other hand, in applications PWM is used as TFBS model
to identify “binding sites” by scanning a given sequence and identifying words with scores
no less than a threshold.

Thus, in reality a TFBS model is related to the set of words
scoring no less than the given threshold for the given PWM. It is desirable to construct a 
similarity measure for TFBS models based on the similarity between word sets recognized
by the matrices with given thresholds, rather than on  similarity between matrices per
se. Moreover, comparison-by-elements strategy requires the matrices to have algebraically 
comparable values (either frequencies or specifically scaled weights) which is not necessary if 
sets of recognized TFBS are compared.

MACRO-APE computes a similarity measure which directly accounts for similarity of
recognized word sets. This measure does not require PWM elements to be algebraically 
comparable and so it can be used to compare weight matrices constructed by different 
normalization / conversion strategies (e.g. log-odds with different pseudocounts and/or background normalization).

Let us have a position weight matrix of length $w$. The whole set of ACGT-alphabet 
words of length $w$ will be called the dictionary of size $N=4^w$. For a fixed threshold level $t$ one 
can calculate the fraction of the dictionary (i.e. the number of words $n$) scoring no less than the 
threshold. We will call the value of $n / N$ as the motif \pvalue.

Suppose we have two PWMs $m_1$, $m_2$ of length $w$ and some \pvalue\ levels $p_1$, $p_2$. For $m_1$
and $m_2$ we can estimate the thresholds $t_1$ and $t_2$ corresponding to $p_1$, $p_2$. Having PWMs with the 
corresponding thresholds we can estimate the fraction f of the dictionary recognized by both 
models, i.e. the size of the set of words scoring no less than $t_1$ on $m_1$ and no less than $t_2$ on $m_2$.

Moreover one can construct the Jaccard index 
$$J = \frac{\mathbf{A}\cap \mathbf{B}}{\mathbf{A}\cup \mathbf{B}}\,, $$
where $\mathbf{A}$ and $\mathbf{B}$ are 
sets of words recognized by $m_1$ and $m_2$ with the thresholds $t_1$ and $t_2$. If necessary one also can 
construct a Jaccard distance as $$d(\mathbf{A},\mathbf{B}) = 1 - J\,.$$

In the general case we have two PWMs of different widths, unknown optimal mutual 
alignment and orientation. For each possible alignment shift and orientation the matrices can be 
extended to the same length by adding zero-columns (not affecting either score or threshold) 
and then compared as the two models of the same width. Then one can determine the optimal 
shift and orientation by selecting the case with the highest Jaccard similarity. More formal and 
detailed explanation can be found in the corresponding macroape paper [MACROAPE].

\textbf{NOTE!} The reverse complementary transformation can be necessary to optimally align a given pair of matrices,
thus the background nucleotide composition for matrix comparison tools should be symmetrical, i.e. p(A) = p(T) and p(C) = p(G).


\subsection{EvalSimilarity}
This is a tool to compute the similarity of two given motifs defined as a Jaccard similarity of sets of words recognized by each motif. Sets of recognized words are given by a PWM accompanied with threshold or a \pvalue. By default set of recognized words is defined as top $0.05\%$ of words (i.e. \pvalue\ level of $0.0005$) ranked by a PWM.

It's possible to set either a different \pvalue\ with \cmdoption{-p \requiredarg{\pvalue}} option or to specify thresholds explicitly so that word sets contain all words passing corresponding thresholds. It can be accomplished using \cmdoption{--first-threshold \requiredarg{threshold}} and \cmdoption{--second-threshold \requiredarg{threshold}} options.

In order to get intuition of Jaccard similarity scale and to better catch our output format, try these examples and take a look at corresponding logos:

\emph{TODO: here aligned motif logos should go}

\exampleof{rather similar motifs \texttt{KLF4\_f2} and \texttt{SP1\_f1}}
\EvalSimilarity{motifs/KLF4\_f2.pat\\ motifs/SP1\_f1.pat}
% \outputheader
% \cmdoutputfromfile{EvalSimilarityOutput(KLF4,SP1).txt}

\exampleof{the same motif \texttt{SP1\_f1} in opposite orientations}
\EvalSimilarity{motifs/SP1\_f1\_revcomp.pat\\ motifs/SP1\_f1.pat}
% \outputheader
% \cmdoutputfromfile{EvalSimilarityOutput(SP1,revcompSP1).txt}

\exampleof{significantly different motifs \texttt{SP1\_f1} and \texttt{GABPA\_f1}}
\EvalSimilarity{motifs/SP1\_f1.pat\\ motifs/GABPA\_f1.pat}
% \outputheader
% \cmdoutputfromfile{EvalSimilarityOutput(SP1,GABPA).txt}


By default \cmdname{EvalSimilarity} tests all possible motif alignments in both orientations. But there is an option \cmdoption{--position} to evaluate similarity in an explicitly specified specified motif alignment. Option usage is following:

\cmdoption{--position \requiredarg{shift},\requiredarg{direct|revcomp}}

Option parameters are comma-separated, spaces not allowed.

Try the following examples:
\exampleof{rather similar motifs \texttt{KLF4\_f2} and \texttt{SP1\_f1} at optimal alignment}
\EvalSimilarity{motifs/KLF4\_f2.pat\\ motifs/SP1\_f1.pat\\ --position -1,direct}
% \outputheader
% \cmdoutputfromfile{EvalAlignmentOutput(RightPosition).txt}

\exampleof{rather similar motifs \texttt{KLF4\_f2} and \texttt{SP1\_f1} at completely wrong alignment}
\EvalSimilarity{motifs/KLF4\_f2.pat\\ motifs/SP1\_f1.pat\\ --position 3,revcomp}
% \outputheader
% \cmdoutputfromfile{EvalAlignmentOutput(WrongPosition).txt}

If your PWMs have predefined thresholds you can specify them explicitly with \cmdoption{--first-threshold \requiredarg{threshold}} and \cmdoption{--second-threshold \requiredarg{threshold}} options. Given a threshold, \cmdname{EvalSimilarity} won't use threshold estimation by a \pvalue. \cmdname{EvalSimilarity} still will calculate Jaccard index of two sets but these sets will be


{\small
\textbf{Note!} By default \cmdname{EvalSimilarity} selects the threshold corresponding to the \pvalue\ not 
less than requested (upper boundary). This is different from \cmdname{FindThreshold} which uses 
lower boundary for \pvalue. 

It is very important to select upper \pvalue\ boundary for short PWMs. In case of given 
low \pvalues\ they can recognize no words at all (so the Jaccard measure may have zero 
numerator and zero denominator). For reasonable threshold levels both upper and lower 
boundaries usually produce very close similarity values. 

Nevertheless, one can override this behavior with \cmdoption{‘--boundary lower’} option. In such a case if 
any of supplied PWMs recognizes no words for a selected \pvalue, then similarity can not be 
correctly determined and macroape will report the similarity value of $-1$. 
}

% The same note is also appliable to the search through a model collection (see below). 

% NOTE! For \cmdname{ScanCollection} tool this affects only a query PWM. The same option should be 
% specified with \cmdname{PrecalculateThresholds} when preprocessing a PWM collection

This tool uses a collection of motifs to find PWMs similar to a given query. 
The list of similar PWMs is sorted by similarity in descending order so the PWMs similar to the query are 
located at the top of the list.

NOTE! The shift and orientation are reported for PWMs from the collection relative to the query 
PWM.

\example{(search for motifs similar to \texttt{KLF4\_f2}, HOCOMOCO collection)}
\ScanCollection{motifs/KLF4\_f2.pat\\ ./hocomoco/}
% \outputheader[Output (STDOUT)]
% \cmdoutputfromfile{ScanCollectionOutput(KLF4,default).txt}

The two-pass search mode is available to recheck the top part of the list using a more precise discretization level. Second pass is executed only 
if \cmdoption{--precise [min\_similarity=0.01]}
key is specified. The precise search will recheck only the PWMs 
similar to the query with a similarity no less than \cmdoption{min\_similarity}. The results of the second pass 
will be marked by asterisk(*).

One can specify similarity cutoff with option \cmdoption{--similarity-cutoff \requiredarg{similarity cutoff}} or \cmdoption{-c \requiredarg{similarity cutoff}} to discard 
comparison results with the resulting similarity less than a given value (the 1st pass results are used). 
By default, records with similarity less than 0.05 are not shown.
In order to print comparison results for all PWMs in collection \cmdoption{--all} option can be used.

\example{(search PWMs similar to \texttt{KLF4\_f2}, extended precision for the most similar PWMs)}
\ScanCollection{motifs/KLF4\_f2.pat\\ ./jaspar/\\ --precise}
% \outputheader[Output (STDOUT)]
% \cmdoutputfromfile{ScanCollectionOutput(KLF4,precise).txt}

To find similar PWMs using a particular \pvalue\ level one should use the \cmdoption{"--pvalue"}~option. Default \pvalue\ is 0.0005.

\exampleof{}
\ScanCollection{motifs/KLF4\_f2.pat\\ ./selex/\\ --pvalue 0.001\\ --similarity-cutoff 0.06 --precise 0.1}
% \outputheader[Output (STDOUT)]
% \cmdoutputfromfile{ScanCollectionOutput(KLF4,CustomPvalue,CustomFilters).txt}



\section{PERFECTOS-APE: Predicting Regulatory Functional Effect of SNPs by Approximate P-value Estimation.}
\subsection{Introduction}
DNA sequence is exposed to mutations. One special and particularly common kind of mutations are single nucleotide substitutions. Genome variants differing by such a mutation are called single nucleotide variants~(SNVs). If some variation became common in population they can be also called single nucleotide polymorphisms~(SNPs).

Being positioned in coding regions of genome, mutations can affect protein sequence, but due to a small size of coding regions across genome, most mutations affect non-coding regions such as intronic and intergenic regions, promoters, 3'- and 5'-UTRs. These substitutions cannot break protein sequence but can interfere gene regulation. One possible mechanism of this process is a change in transcription factor~(TF) affinities to a mutated nucleotide sequence. Substitution can disrupt or create a transcription factor binding site~(TFBS). PERFECTOS-APE software is designed to evaluate difference in transcription factor affinity for different SNVs.

PWM corresponding to a transcriptiopn factor provides a score for a putative TFBS. This score represents binding affinity, thus big difference in scores means big difference in binding affinities. Though as was discussed earlier [ref to APE section] scores are not comparable directly because PWM scores can be arbitrarily scaled. More convenient measure of affinity is a \pvalue\ - the probability to find a sequence with score not lower than given.

PERFECTOS-APE compares TF affinity \pvalues\ for each sequence variant and calculates fold change of a substitution. Detailed algorithm for evaluating a fold change for a given TF and a substituion:

\begin{itemize}
\item For each genome position such that corresponding TFBS overlaps mutation position calculate PWM scores for each sequence variant.
\item Independently choose the best position and score for both sequence variants.
\item Estimate \pvalues\ for the best scores.
\item Fold change is the rate of \pvalues.
\end{itemize}

PERFECTOS-APE tool tests each of given SNVs against a whole collection of TFs and yields (SNV, TF) pairs for which SNV significantly affected TF affinity to a binding site. Tool filters out such SNVs that TF doesn't have a site in any of sequence variants and SNVs which have fold change close to $1.0$.

For score to \pvalue\ transformation we use two different approaches. First approach follows Touzet paper []. Second approach uses list of precalculated (score, \pvalue) sample pairs and uses binary search to find an approximate \pvalue. Precalculation strategy can be very useful taking into account that usually PWM collection is not a subject to change (and is strongly recomended for processing of tens or more SNVs and for working with dinucleotide motifs).


\cmdname{SNPScan} takes a list of SNVs with flanking sequences and a motif collection and returns a list of predicted TFBS which were possibly disrupted by or emerged after a certain SNV.
If flanking sequence is too short it's padded with poly-N tail up to necessary length.
\usageheader
\SNPScan{\requiredarg{path to the collection of motifs} \requiredarg{path to the file with the list of SNVs} \optionalarg{options}}


\cmdname{SNPScan} has two filters. The first discards (SNV,~TF) pairs without TFBS prediction at any of nucleotide variants. \cmdname{SNPScan} treat a word as a putative TFBS if \pvalue\ of this word's score is not greater than the predefined threshold (0.0005 by default, changed via \cmdoption{--pvalue-cutoff} option:\\
\cmdname{--pvalue-cutoff \requiredarg{maximal \pvalue\ to be considered}}\\
or in short form:\\
\cmdname{-P \requiredarg{maximal \pvalue\ to be considered}}.

The second filter requires check \pvalue\ fold change to be large enough. By default fold change threshold is equal to $5$. It means that only SNVs causing \pvalue\ change of 5x and more ($FoldChange \ge 5$ or $FoldChange \le 1/5$) will be included in results. Fold change threshold can be specified using \cmdoption{--fold-change-cutoff}:\\
\cmdoption{--fold-change-cutoff \requiredarg{minimal fold change to be considered}}\\
or in short form:\\
\cmdoption{-F \requiredarg{minimal fold change to be considered}}

\cmdoption{--log-fold-change} option changes fold change from $\frac{\pvalue{}_1}{\pvalue{}_2}$ into
$\log_{2}{\frac{\pvalue{}_1}{\pvalue{}_2}}$ both in command-line parameter settings and output.

Option \cmdoption{--expand-region \requiredarg{length}} allows PWM hits to be located nearby but not strictly overlap the position with the nucleotide substitution.

When this option is specified, the PWM occurrence can be located up to \texttt{length} bp away from the SNV position.

This option is intended for analysis involving control data with SNVs not necessarily overlapping the binding sites.


The last but the most useful option is \cmdoption{--precalc} which forces \cmdname{SNPScan} to work with precalculated \pvalue,thresholds pairs performing binary search to evaluate the \pvalue\ instead of calculating motif score distribution each time from scratch. It can reduce total computation time in hundreds of times for large datasets.
As an input it requires a folder with precalculated (\pvalue,threshold) pairs - one for each motif:\\
\cmdoption{--precalc \requiredarg{path to a folder with precalculated \pvalue, threshold pairs}}

These precalculated score distributions are to be produced by a \cmdname{PreprocessCollection} from APE toolbox.
Please refer to the respective section for details.


\example{}
\SNPScan{./hocomoco/pwms/ snp.txt --precalc ./collection\_thresholds}
\SNPScan{./hocomoco/pcms/ snp.txt --pcm --discretization 10 --background 0.2,0.3,0.3,0.2}

\subsubsection{Output data format}

\cmdname{SNPScan} prints all results to standard output, errors and messages go into standard error stream. First line of output is a header of table. Latter lines are rows of this table. Columns are:
\begin{itemize}
\item Name of sequence containing SNV
\item TF motif name
\item for the first allele variant:
\begin{itemize}
\item   the best position and strand of putative TF-DNA binding
\item   nucleotide word corresponding to the best binding sequence among all other words in sequence, intersecting SNV
\end{itemize}
\item the same two columns for the second allele variant
\item allele variants
\item \pvalue\ for the first allele variant
\item \pvalue\ for the second allele variant
\item fold change (the first \pvalue\ divided by the second \pvalue)
\end{itemize}

Position of the best binding place is given for the leftmost boundary of a binding sequence (independent of strand orientation). The SNV location is at zero, so the TFBS coordinates are always less or equal to zero. Strand is denoted as `direct` or `revcomp`. Words are given at the relevant strand (i.e. reverse-complement transformation is applied if necessary).

More compact output format can be produced using the \cmdoption{--compact} option.

The resulting table will have the following columns:
\begin{itemize}
\item Name of sequence containing SNV
\item TF motif name
\item \pvalue\ for the first allele variant
\item \pvalue\ for the second allele variant
\item the best position and strand of putative TF-DNA binding for the first allele variant
\item the best position and strand of putative TF-DNA binding for the second allele variant
\end{itemize}

Please note that fold change and word sequences are not shown (comparing to the default output).
Strand information is given as +/- form (versus direct/revcomp in the default output).
\pvalues are rounded up to three significant digits.

This option is intended to process huge lists of SNVs and reduce the output (~2.5x less size).



\section{Data formats}
% Augment with information about transposed matrices with a link to options
\subsection{Position matrix format description}
All tools in the *-APE package use the following matrix file format (each binding site position 
corresponds to a separate line):

\texttt{\begin{tabular}{llll}
some\_header\\
pos1\_A\_weight & pos1\_C\_weight & pos1\_G\_weight & pos1\_T\_weight\\
\ldots\\
posw\_A\_weight & posw\_C\_weight & posw\_G\_weight & posw\_T\_weight
\end{tabular}
}

Position matrix format is appliable for all kinds of positional matrices: positional weight(PWM), count(PCM) and probability/frequency(PPM).
Positonal count matrices are allowed to contain floating point numbers (e.g. in the case the counts were derived from 
somehow weighted alignments).

The total number of lines corresponds to the PWM width (minus the header line). If given, header will be treated as a motif name, otherwise filename will stand for motif name. Header may carry an optional \texttt{"\textgreater"} sign at line start
(like in fasta files).

If necessary it's possible to read transposed matrices, with nucleotides in rows and positions in columns using \cmdoption{--transpose} option.

\exampleof{PWM similar to HOCOMOCO transcription factor motif for KLF4}
{\small\verbatiminput{./MotifSamples/KLF4_f2_alike.pwm}}

\exampleof{Transposed PWM similar to HOCOMOCO transcription factor motif for KLF4}
{\small\verbatiminput{./MotifSamples/KLF4_f2_alike_transposed.pwm}}

More real-life examples are provided with the package in respective motif collections.

Dinucleotide versions of *-APE tools use dinucleotide motifs. 
Dinucleotide positional matrices have similar format but contain 16 columns instead of 4. Columns go in order: AA, AC, AG, AT, CA, CC, \dots, TT. It's also possible to use mononucleotide motifs in dinucleotide tools (e.g. to use dinucleotide background). For rationales and details take a look at \cmdoption{--from-mono} option.

\cmdname{SNPScan} uses a list of sequences with SNVs as input data.

The list of sequences with SNVs should be given in a single plain text file. 
Each sequence should be presented at a separate line using the following format:\\
\texttt{\inangles{SNV name} \inangles{left flank}[\inangles{variant 1}/\inangles{variant 2}]\inangles{right flank}}

SNV name shouldn't contain empty delimiters (spaces or tabs). 
Sequence consists of two allele variants in square brackets, separated with `/`, and flanking sequences at both sides. Length of flanking sequences should be sufficient to place the longest motif of a given collection (so it is advised to provide 25-30bp at each side) into all positions relative to a nucleotide substitution position. 

So, first two columns are SNV name and SNV sequence. Later columns (if present) are ignored, thus can contain any data.

\exampleof{SNV list}
\noindent\texttt{%
\# Text after "\#" doesn't matter\\
\# It's possible to include any number of comment lines into input\\
rs10040172 gattgcagttactga[G/A]tggtacagacatcgt\quad Anything\\
rs10116271 gtggggaagaggtct[C/T]gtagaggcgatgatt\quad can go\\
rs10208293 ttatgtccagtacct[A/G]tggaccctccttgtg\quad after first\\
rs10431961 ggtcaggcggcgtcg[C/T]cggtacgctctgagc\quad two columns
}

\vspace{1.5ex}
Note that lines starting with \texttt{\#} are considered as comments and thus ignored by \cmdname{SNPScan}.



\section{Additional command-line options}
Each tools can be configured with multiple options. All options should go after the required arguments. There are some options common for the majority of MACRO-PERFECTOS-APE tools and some options are tool specific. Tool-specific options were already described in the Tools section. Common options will  be described in this section.

Options can be divided into two large classes: those which alter input data format and those which affect tool calculation. The first class of options allows one to load motifs from PCM or PPM instead of PWM, read matrices in non-standard transposed format and make mononucleotide motifs mimic dinucleotide ones. The second class of options allow specifing background model, select \pvalue\ evaluation mode, limit memory consumption and so on.

Some tools have multiple options correspondning to the same option type. In such cases this manual describes the only one option from the family. This manual doesn't describe each of options separately and doesn't try to list all the options the tool. For a full list of tool options, please use \cmdoption{--help} command line option.

\subsection{Option families}
{\small
Families of options have similar names, but different prefices. For example \cmdname{macroape.di.EvalSimilarity} tool, has an option \cmdoption{--from-mono}. This option means that both dinucleotide motifs should be obtained from mononucleotide ones. In turn \cmdoption{--first-from-mono} options means that the first motif will be obtained from mononucleotide input. \cmdoption{--second-from-mono} does the same but for the second motif.

Analogous options for \cmdname{macroape.ScanCollection} are named \cmdoption{--query-from-mono} and \cmdoption{--collection-from-mono}. Option \cmdoption{--query-from-mono} demands mononucleotide query motif, and \cmdoption{--collection-from-mono} means that each motif in collection should be loaded from mononucleotide motif. The same is appliable for options like \cmdoption{--background}.

Such triples of options are typically listed in the help string like this: \cmdoption{--[first-|second-]from-mono}. It means that one can use both prefixed and non-prefixed options. And possible prefixes are given in square brackets separated with a pipe sign \texttt{"|"}.

\textbf{Note:} Be careful! Prefixed options are always written in a long form. One can use both \cmdoption{-b} and \cmdoption{--background} as synonymous but even if \cmdoption{--first-background} is available there is no \cmdoption{--first-b} option.

\textbf{Note:} Existence of separate options for each motif doesn't necessarily involve existence of common option. E.g. \cmdname{macroape.EvalSimilarity} has options \cmdoption{--first-threshold} and \cmdoption{--second-threshold} but doesn't have \cmdoption{--threshold} because it has no sense.
}

\subsection{Option families}
{\small
Families of options have similar names, but different prefices. For example \cmdname{macroape.di.EvalSimilarity} tool, has an option \cmdoption{--from-mono}. This option means that both dinucleotide motifs should be obtained from mononucleotide ones. In turn \cmdoption{--first-from-mono} options means that the first motif will be obtained from mononucleotide input. \cmdoption{--second-from-mono} does the same but for the second motif.

Analogous options for \cmdname{macroape.ScanCollection} are named \cmdoption{--query-from-mono} and \cmdoption{--collection-from-mono}. Option \cmdoption{--query-from-mono} demands mononucleotide query motif, and \cmdoption{--collection-from-mono} means that each motif in collection should be loaded from mononucleotide motif. The same is appliable for options like \cmdoption{--background}.

Such triples of options are typically listed in the help string like this: \cmdoption{--[first-|second-]from-mono}. It means that one can use both prefixed and non-prefixed options. And possible prefixes are given in square brackets separated with a pipe sign \texttt{"|"}.

\textbf{Note:} Be careful! Prefixed options are always written in a long form. One can use both \cmdoption{-b} and \cmdoption{--background} as synonymous but even when \cmdoption{--first-background} is available there is no \cmdoption{--first-b} option.

\textbf{Note:} Existence of separate options for each motif doesn't necessarily involve existence of common option. E.g. \cmdname{macroape.EvalSimilarity} has options \cmdoption{--first-threshold} and \cmdoption{--second-threshold} but doesn't have \cmdoption{--threshold} because it has no sense.
}


\subsubsection{Dinucleotide motifs format}
Dinucleotide version of tools work with dinucleotide motifs. Dinucleotide positional matrices have similar format but contain 16 columns instead of 4. Columns go in order: AA, AC, AG, AT, CA, CC, \dots, TT.

It is also possible to use mononucleotide PWMs instead of dinucleotide PWMs. Conversion PWM~$\rightarrow$~DiPWM will be done internally in such a way that each word has the same score on DiPWM as it had on PWM. It's done with \cmdoption{--from-mono} option. 

\cmdoption{--first-from-mono} and family (analogous to options described in \ref{TransposeOption} section) are also available.

Obtaining dinucleotide motifs from mononucleotide can help you to compare mononucleotide PWM with dinucleotide PWM. Another reason to use them - is a study of mononucleotide motif properties on dinucleotide background.

{\small\textbf{Notice:} scores of words on discreted PWM and corresponding diPWM can be slightly different due to a discretization step performed after PWM~$\rightarrow$~DiPWM conversion. This discrepancy shouldn't worry you, it's small enough and goes to zero with discretization increase.}


% Here we should insert some link to pcm-pwm conversions
\subsubsection{\cmdoption{--transpose} option for motifs specified in different orientation}\label{TransposeOption}
You can use motifs with nucleotides in rows, by specifying \cmdoption{--transpose}. The only difference in format is matrix orientation, header remains the same. This option is available for each tool.

Macroape tools work with pairs of motifs and \cmdoption{--transpose} option mean that both motif sources are transposed. In tool \texttt{macroape.EvalSimilarity} one can use \cmdoption{--first-transpose} or \cmdoption{--second-transpose} instead to use one matrix specified in horizontal orientation, another in vertical. Similar options for \texttt{macroape.ScanCollection} are named \cmdoption{--collection-transpose} and \cmdoption{--query-transpose}.


\subsection{Background model options}
Nucleotide frequencies of a background model can be specified in optional arguments, e.g. \cmdoption{-b} or \cmdoption{--query-background}. All background options use the same format with a single required argument: \cmdoption{-b \requiredarg{value}}.

Default background model is a \texttt{wordwise} model. It means that all our calculations assume uniform nucleotide distribution and the exact number of words is used everywhere instead of probabilities of a word set.
E.g. \texttt{FindPvalue} will calculate not the probability of a random word score to pass the threshold but a fraction of words scoring greater than threshold estimating the exact number of such words.

A number of words is a more natural and intuitive to use, especially if the background model cannot be properly selected
thus we suggest "wordwise" mode by default.

Wordwise mode can be specified explicitly, e.g. using \cmdoption{-b wordwise} key.

All following formats are different ways to specify frequencies of each nucleotide:
\begin{itemize}
\item The most simple nucleotide background model is uniform, each nucleotide has the same probability to occur. Option format is: \cmdoption{-b uniform}. This is close to wordwise mode, but word set probabilities are used and reported instead of raw counts of words.
\item It is also possible to specify a fixed GC-content~(in range 0 to 1): \cmdoption{-b \requiredarg{GC-content}}. E.g.~\cmdoption{"-b~0.6"}
\item The most detailed format is to explicitly specify nucleotide frequencies: \cmdoption{-b \requiredarg{$p_A,p_C,p_G,p_T$}}. E.g.~\cmdoption{"-b 0.2,0.3,0.3,0.2"} will define the same frequencies as for GC-content of 0.6. Note that nucleotide frequencies should be given in alphabetical ACGT-order separated with commas. 
\end{itemize}

\textbf{Note:} No spaces between frequencies are allowed (commas only). Sum of frequencies should be equal to 1.0.

\subsubsection{Dinucleotide background}
Dinucleotide background has the same options: wordwise, uniform, GC-content and dinucleotide frequencies. Worwise, uniform and GC-content backgrounds are effectively the same as mononucleotide ones and don't carry nucleotide interdependencies.

Dinucleotide frequencies in turn require some additional notes. Dinucleotide frequencies go in standard dinucleotide order: AA, AC, AG, \dots, TT~--- 16 terms. Each value means probability of a specific dinucleotide. These probabilities are \textbf{not} conditional probabilities used by an algorithm, conditional probabilities are calculated internally. Be careful if you got used to use Markov model background.

Again, list of frequencies is comma-separated, no spaces allowed, sum of probabilities should be equal to 1.0.

Also one can specify mononucleotide ACGT-frequencies background even in dinucleotide tools. It will be recognized automatically when 4 values are specified instead of 16 ones. In such a case dinucleotide background will treat nucleotide as non-depending from previous one.


\subsubsection{Specifying custom discretization level}
For a more precise result \cmdoption{--discretization \requiredarg{discretization rate}} or \cmdoption{-d \requiredarg{discretization rate}} command line key can be used to explicitly set
the discretization level for PWM elements, like \cmdoption{"--discretization 100000"} (see the section \ref{discretization-strategy}).
The discretization level of $10^5$ corresponds to the precision of PWM elements up to 5 decimal places.
A larger number of decimal places results in increased precision and computational time.
The default setting of $10^4$ for single-motif tools and $10^1$ for motif comparison tools gives reasonable "time-precision" tradeoff.

\subsubsection{Specifying custom \pvalue\ level}
All tools in \texttt{MACRO-APE} package estimate motif threshold by a \pvalue\ for further use. By default \pvalue\ level of 0.0005 is assumed.
It can be overriden with \cmdoption{--pvalue \requiredarg{\pvalue}} or \cmdoption{-p \requiredarg{\pvalue}} option key.

\subsubsection{Choose proper threshold by a \pvalue}
All *-APE tools except \cmdname{ape.FindPvalue} and \cmdname{perfectosape.SNPScan} perform internal \pvalue\ to threshold conversion.
Since PWM \pvalues\ have discrete distribution a given \pvalue\ can be achieved only approximately.
A fixed threshold corresponds to the actual \pvalue\ which is smaller or larger than the requested \pvalue.

The boundary selection can be done using \cmdoption{--boundary \requiredarg{lower|upper}}.

For model comparison by default we use the upper boundary for the \pvalue~(so even at low given \pvalues\ PWMs recognize some words and thus the models can be 
compared). If searching for a threshold corresponding to the given \pvalue\ we report the lower boundary of the \pvalue\ by default (to properly control
the positive prediction rate corresponding to a given threshold).

\textbf{Note:} \texttt{lower} boundary means that \pvalue\ will be not greater than the requested one.
The threshold for \texttt{lower} \pvalue\ will be greater than the threshold for \texttt{upper} boundary \pvalue.


\subsubsection{Limiting CPU and memory consumption}
It's possible to create an artificially arranged PWM whose score distribution
 will grow exponentially with length and thus can take a lot of memory and time for computation.
 This option is mostly designed to prevent *-APE tools from unnormal CPU and memory consumption.
 If hash size exceeded a given limit, tools cancel calculations with \texttt{"Hash overflow"} error message.
 In such case user can manually expand hash size limits or lower discretization level.
  \begin{itemize}
  \item \cmdoption{--max-hash-size \requiredarg{size}}: set the internal hash (used for score distribution calculation) size limit. Default value is $10^7$
  \item \cmdoption{--max-2d-hash-size \requiredarg{size}}: set the internal two-dimensional hash size limit (used for PWM comparison in \texttt{MACRO-APE} toolbox). Default value is $10^4$.
  \end{itemize}



\section{Formal math}
  Following the general idea described in \cite{Touzet2007} we can effectively calculate the \pvalue\ 
for a given PWM with a fixed precision and a given threshold value. The algorithm of Touzet \textit{et al.}
efficiently processes matrices with integer elements. The matrices with real values are transformed
into integer value matrices by multiplying each value by discretization constant and truncating the decimals.

Effectively this is similar to rounding up real values leaving only the fixed number of
decimal places. The higher discretization level will result in a more accurate \pvalue\ calculation 
and an increased computational time. 

Please note, that in contrast to the original Touzet algorithm here we applying "ceil" 
operation to the matrix elements (instead of "floor" in the original paper of Touzet). This allows 
us to have a strict upper boundary of the threshold for a given \pvalue.

We use the default discretization level of $10^4$ to perform calculations with accuracy up to
four significant digits for single-PWM tools from APE toolbox.

For motif comparison the straightforward discretization by rounding up to
the nearest integer is used by default for a fast and rough search through the motif collection. 
The default level of $10$ (one decimal place) is used for a more precise search of similar motifs.

Thus in our case discretization is the transformation as follows: discretized~$S$ is
$S$~multiplied by discretization level~$V$ and rounded up to the nearest integer value.

\texttt{
\begin{tabular}{ll}
Example:\\
S = 1.6734\\
discretization V=1 & discretized S = $\lceil 1.6734 \rceil$ = 2\\
discretization V=10 & discretized S = $\lceil 16.734 \rceil$ = 17\\
discretization V=100 & discretized S = $\lceil 167.34 \rceil$ = 168
\end{tabular}
}

Discretization will generally preserve the word score ranking with the common exception 
for words that would obtain identical scores. The main advantage of the discretization 
is decreasing of the number of possible scores so the set of all possible scores can be 
enumerated more effectively.

  \subsubsection{Obtaining PWM from PCM and PPM models}
One can specify motifs in PCM or PPM formats. If so, additional step of PCM~$\rightarrow$~PWM or PPM~$\rightarrow$~PWM conversion to be performed. In such a case \cmdoption{--pcm} or \cmdoption{--ppm} should be specified.

Matrix of positional counts (PCM) can be transformed into PWM according to the formula used in [Lifanov2003]:
$$ PWM_{\alpha,j} = \ln\frac{ PCM_{\alpha,j} + aq_{\alpha} }{ (W+a)q_{\alpha} }\,, $$

where $\alpha$ is a nucleotide index and $j$ is a position index; $W$ is the total weight of the alignment (or the number of aligned words), $a$ is the 
pseudocount value selected by default as the $\ln(W)$, and $q_{\alpha}$ is the background probability of nucleotide letter $\alpha$.

This conversion procedure isn't the only one possible, but all our tools use this procedure. If you need another one, you can manually convert PCMs into PWMs according to your owm algorithm.

This algorithm has several parameters. At first, pseudocount $a$ which has a default value of $\ln(W)$ can be redefined. In order to set pseudocount to a constant, one can use \cmdoption{"--pseudocount \requiredarg{value}"} option.

PPM~$\rightarrow$~PWM conversion is done in two stages. First PPM is multiplied by a constant alignment weight $W$ to obtain a PCM. Then this PCM is converted to a PWM as described above. 
A user should supply alignment weight explicitly by the \cmdoption{--effective-count \requiredarg{alignment weight}} option. If this information is not given, alignment weight of 100.0 will be used as a default assumption.

Backgrounds specified for tool-specific purposes will be also used for PCM~$\rightarrow$~PWM conversion.

\emph{TODO: Dibackground motifs conversion strategy}


\section*{References}
[MACROAPE] Algorithms Mol Biol. 2013, 8:23.
Jaccard index based similarity measure to compare transcription factor binding site models.
Ilya E Vorontsov, Ivan V Kulakovskiy and Vsevolod J Makeev

[Touzet2007] Algorithms Mol Biol. 2007 Dec 11;2:15. Efficient and accurate \pvalue  
computation for Position Weight Matrices. Touzet H, Varré JS.

[Pape2008] Bioinformatics. 2008 Feb 1;24(3):350-7. Epub 2008 Jan 2. Natural similarity 
measures between position frequency matrices with an application to clustering. Pape UJ, 
Rahmann S, Vingron M.

[Lifanov2003] Genome Res. 2003 Apr;13(4):579-88. Homotypic regulatory clusters in 
Drosophila. Lifanov AP, Makeev VJ, Nazina. AG, Papatsenko DA.
\end{document}
