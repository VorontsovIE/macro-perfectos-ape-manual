\documentclass[draft]{article}

\usepackage{amssymb}
\usepackage{amstext}
\usepackage{amsmath}
\usepackage{graphicx}
\usepackage[colorlinks=true, urlcolor=blue, pdfborder={0 0 0}]{hyperref}
\usepackage{verbatim}

\tolerance=500
\newcommand*{\urldecorated}[1]{[\url{#1}]}

\newlength{\firstcolwidth}

\newcommand*{\secondcolindent}[3][2em]{
\par
{
\parindent=#1
\settowidth{\firstcolwidth}{#2}
\addtolength{\firstcolwidth}{#1}
\setlength{\hangindent}{\firstcolwidth}
#2#3
\par
}
}

\newcommand*{\ProgramInvokation}[2]{
\secondcolindent[2em]{\texttt{java -cp ape.jar \mbox{ru.autosome.#1} }}{\texttt{#2}}
}

\newcommand*{\DefineProgramInvokationCommand}[2]{
\newcommand{#1}[1]{
\ProgramInvokation{#2}{##1}
}
}

\DefineProgramInvokationCommand{\EvalSimilarity}{macroape.EvalSimilarity}
\DefineProgramInvokationCommand{\ScanCollection}{macroape.ScanCollection}
\DefineProgramInvokationCommand{\FindThreshold}{ape.FindThreshold}
\DefineProgramInvokationCommand{\FindPvalue}{ape.FindPvalue}
\DefineProgramInvokationCommand{\PrecalculateThresholds}{ape.PrecalculateThresholds}
\DefineProgramInvokationCommand{\SNPScan}{perfectosape.SNPScan}


\newcommand*{\example}[1]{\par\vspace{2ex}\noindent\textbf{Example#1:}\par\nopagebreak\vspace{0.5ex plus 0.3ex minus 0.1ex}}
\newcommand*{\exampleof}[1]{\example{ (#1)}}
\newcommand*{\usageheader}[1][Usage]{\par\noindent\textbf{#1:}\par\nopagebreak}
\newcommand*{\outputheader}[1][Output]{\par\noindent\textbf{#1:}\par\nopagebreak}
\newcommand*{\cmdoutput}[1]{{\small#1}}
\newcommand*{\cmdoutputfromfile}[1]{\cmdoutput{\verbatiminput{./ProgramOutput/#1}}}


\newcommand*{\inangles}[1]{\textless#1\textgreater}
\newcommand*{\requiredarg}[1]{\inangles{#1}}
\newcommand*{\optionalarg}[1]{[#1]}
\newcommand*{\cmdname}[1]{\mbox{\texttt{#1}}}
\newcommand*{\cmdoption}[1]{\mbox{\texttt{#1}}}

\newcommand*{\pvalue}{\mbox{P-value}}
\newcommand*{\pvalues}{\mbox{P-values}}

\begin{document}

\title{
MACRO-PERFECTOS-APE~---\\{\small MAtrix CompaRisOn \&\\ Predicting Regulatory Functional Effect of SNPs\\ by Approximate \pvalue\ Estimation}\\
-- User Manual --
}
\maketitle

\section{Abstract}
Here we present MACRO-APE and PERFECTOS-APE software designed for practical sequence analysis involving classic mononucleotide and dinucleotide position weight matrices (PWMs) of DNA sequence patterns often called \textit{motifs}. The common usage case for DNA motifs is representation of transcription factor binding sites.

The software allows (1) comparing different PWMs using a variant of Jaccard similarity measure, e.g. scanning a motif collection for motifs similar to a given query, (2) analysing single-nucleotide variants for possible regulatory effect through transcription factor affinity changes, (3) performing basic PWM analysis (\pvalue\ and threshold estimation).


\section{Technical notes}
MACRO- and PERFECTOS-APE require Java Runtime Environment 1.6 (or newer) to run. It works under Windows and Linux 
environments. 

HOCOMOCO-AD \urldecorated{http://autosome.ru/HOCOMOCO} TFBS model collection and several examples of PWMs (motifs) can be downloaded with MACRO-PERFECTOS-APE at \emph{Link to motif collections should be here!}.

Windows users can get the latest Java directly: \urldecorated{http://www.java.com/en/download/manual.jsp}. 
Modern Linux distributions usually already has it installed.

The latest MACRO-PERFECTOS-APE package is always available at \urldecorated{http://opera.autosome.ru/}. Source codes are distributed under GNU GPL and DO WHAT THE FUCK YOU WANT TO public licenses. They are available in a github repository: \urldecorated{https://github.com/prijutme4ty/macro-perfectos-ape}.


\section{Overview}
All tools are packed in a jar-file with compiled Java classes. There are three main packages for tools:
\texttt{ru.autosome.ape}, \texttt{ru.autosome.macroape} and \texttt{ru.autosome.perfectosape}.

\textbf{APE} in \texttt{ru.autosome.ape} stands for Approximate \pvalue\ Estimation, this package contains basic tools:
\begin{itemize}
\item \cmdname{FindThreshold}~--- to estimate a PWM score threshold for a given \pvalue
\item \cmdname{FindPvalue}~--- to estimate a PWM \pvalue\ corresponding to a given score threshold
\item \cmdname{PrecalculateThresholds}~--- to precalculate lists of thresholds tabulated by \pvalues\ for a given motif collection
\end{itemize}

\textbf{MACRO-APE} in \texttt{ru.autosome.macroape} denotes MAtrix CompaRisOn by Approximate \pvalue\ Estimation. Package consists of several tools related to motif comparison:
\begin{itemize}
\item \cmdname{EvalSimilarity}~--- to evaluate similarity for a given pair of PWMs.
\item \cmdname{ScanCollection}~--- to search a collection of motifs for PWMs similar to a given query.
% \item \cmdname{CollectDistanceMatrix}~--- to collect pairwise distance matrix for a motif collection which can be further used for motif clustering. (Not yet finalized tool!)
% \item \cmdname{AlignMotifs}~--- to evaluate the best motif relative alignment. (This tool not yet moved from ruby to Java!)
\end{itemize}

\textbf{PERFECTOS-APE} in \texttt{ru.autosome.perfectosape} denotes Predicting Regulatory Functional Effect of SNPs by Approximate \pvalue\ Estimation. Package contains a single tool:
\begin{itemize}
\item\cmdname{SNPScan}~--- to search a pack of sequences with SNVs or SNPs against a collection of PWMs for (SNV, PWM) pairs, such that single nucleotide substitution induces significant change of predicted affinity for a given PWM.
\end{itemize}

\textbf{Please note}, that *-APE tools by default consider all given matrices as positional weight matrices with additive scores already passed counts-to-weights transformation (e.g. log-odds). The usage of count matrices (PCMs) or frequency matrices (PPMs) is also possible with additional command-line keys (see the respective section).

\subsection{Command line format}
All tools follow common usage format which is to be described below. We suggest you have a package in a file named \texttt{ape.jar} and located in the current folder. Then typical call of a tool will look like:\par
\texttt{\secondcolindent{java -cp ape.jar }{ru.autosome.ToolName\\ \requiredarg{required arguments}\\ \optionalarg{options}}}

Each command being used with \cmdoption{--help} or \cmdoption{--help} option prints help string describing order of arguments and a list of optional arguments for each command.

Each tools is duplicated for working with mononucleotide and dinucleotide PWMs and corresponding background models. If not sure, use mononucleotide version. Naming convention is the same for all tools: mononucleotide version is located in package's root, dinucleotide version has the same name but is located in a subpackage \texttt{".di"}.

E.g. for \cmdname{ape.FindThreshold} tool, class names are:
\begin{itemize}
\item\cmdname{ru.autosome.ape.FindThreshold} for mononucleotide version
\item\cmdname{ru.autosome.ape.di.FindThreshold} for dinucleotide version.
\end{itemize}

Obviously, dinucleotide versions of tools use different input formats of Position Weight Matrices (DiPWM) and backgrounds. Input data formats will be discussed later.


\subsubsection{Output formats}

All tools except \cmdname{PrecalculateThresholds} print results into stdout (standard output stream).
\cmdname{PrecalculateThresholds} yields result to a set of output files in a specified folder. 

\cmdname{PrecalculateThresholds} and \cmdname{ScanCollection} print progress information to the stderr stream. This output can be disabled with \cmdoption{--silent} option.

Output generally consists of two line types. Lines starting with \texttt{"\#"} character (commented out lines) show 
input parameter values and descriptions. The results are presented in non-commented lines.


\section{Basic APE tools}
APE tools generally deals with a task of converting motif thresholds to \pvalues\ and vice versa.

Position weight matrix~(PWM) of a transcription factor~(TF) assigns a score to each word (nucleotide sequence of fixed length). It makes possible to range the words by their affinity to a TF. Given a threshold, one can divide all words into two subsets: words whose score exceeds the threshold and the rest. First subset represents putative transcription factor binding sites~(TFBS).

It's a common task to evaluate a \pvalue\ of a certain PWM score, i.e. a probability to generate a word at random with the score not less than given. It provides an uniform measure to assess a TF affinity to a binding site.

Inverse task is obtaining a threshold by a \pvalue. The most common use case is assigning a threshold for a PWM such that a constant rate of words are sites (e.g. top 0.05\%).

Our tools perform threshold -- \pvalue\ conversion implementing a dynamic programming algorithm on a discreeted PWM models, similar to one described in [Touzet].


\subsection{FindThreshold}
Stand-alone tool to search for the threshold corresponding to a given \pvalue\ for a given PWM. 
This scripts requires a PWM and a \pvalue\ as input and returns a threshold for which the set of 
words scoring with this PWM above the given threshold has the aggregated probability equal to 
the given \pvalue. The program can process a set of \pvalues, and return a set of thresholds. 
This tool implements the algorithm similar to that implemented in TFM-Pvalue software of 
Helen Touzet \urldecorated{http://bioinfo.lifl.fr/TFM/TFMpvalue/} with fixed discretization level (see "PWM 
discretization" section below).

\usageheader
\FindThreshold{\requiredarg{motif file}\\ \optionalarg{list of \pvalues}}

\exampleof{motif file \texttt{KLF4\_f2.pat}, \pvalue\ of 0.001 and 0.0005}
\FindThreshold{motifs/KLF4\_f2.pat 0.001 0.0005}
% \outputheader
% \cmdoutputfromfile{FindThresholdOutput.txt}

\emph{TODO: to be moved to a discretiztation section}\par
For a more precise result one can use \cmdoption{"-d \requiredarg{discretization rate}"} command line key like \cmdoption{"-d 100000"} to explicitly set 
the discretization level for PWM elements (see the "PWM discretization" section below for 
details). The discretization level of $10^5$ corresponds to the precision of PWM elements up to 
5 decimal places. A larger number of decimal places results in the increased precision and 
computational time. The default setting of 10000 gives reasonable "time-precision" tradeoff.

NOTE! By default \cmdname{FindThreshold} looks for threshold large enough to obtain \pvalue\ not 
greater than requested (lower boundary for \pvalue). For details see \cmdoption{--boundary} option description.

\cmdname{FindPvalue} is a stand-alone tool to find the \pvalue\ corresponding to a given threshold level for a given PWM.

\usageheader
\FindPvalue{\requiredarg{motif file}\\ \requiredarg{list of thresholds}}

\exampleof{motif file \texttt{KLF4\_f2.pat}, thresholds of 4.1719 and 5.2403}
\FindPvalue{motifs/KLF4\_f2.pat 4.1719 5.2403}
% \outputheader
% \cmdoutputfromfile{FindPvalueOutput.txt}

\subsection{PrecalculateThresholds}
This tool is intended to process the motif collection (presented as a folder containing separate 
files for each motif) and to store precomputed score distributions of motif PWMs. Each score distributions is saved as a sorted list of \mbox{(threshold,\pvalue)} pairs taken uniformely as score distribution percentiles. It allows for faster score -- \pvalue\ conversion performing binary search through a list of thresholds or \pvalues. \cmdname{PrecalculateThresholds} doesn't store precise score distribution because for a non-disretized PWM it can be extremely large due to unnecessary precision. In practice it's sufficient to be able to estimate \pvalue\ with a specified error level of e.g. 5\%.

In order to use precalculated distribution most tools have \cmdoption{--precalc} option which takes a folder with thresholds. See corresponding section in option descriptions.

\textbf{Note:} Score distribution precalculation allows one to notably (in hundreds of times) increase speed of threshold to \pvalue\ calculation. Unfortunately it deals with a file system and demands loading of one more file. That's why usage of precalculated score distribution can slow down a bit some tools. It's recomended to use precalculated score distribution for tasks where the same motif \pvalue\ evaluation is performed multiple times so that the score distribution is loaded once and used multiple times. At a moment there is the only use case~-- \cmdname{perfectosape.SNPScan} which assesses each of multiple SNPs against the same motifs.
% TODO: yet undocumented CollectDistanceMatrix also can take an advantage of PrecalculateThresholds (but there are more effective approaches).

\textbf{Note:} Resulting score distribution depends on a discretization level and on a specified background model. Be careful and make sure that a score distribution was precomputed with the same parameters as another tool would use.

\usageheader
\PrecalculateThresholds{\requiredarg{motif collection folder}\\ \requiredarg{folder for results output}\\ \optionalarg{options}}

\example{}
\PrecalculateThresholds{./motifs/ ./motif\_thresholds/}

This will create \texttt{./motif\_thresholds/} folder (if not already exist) and multiple files, one for each motif in \texttt{./motifs/} folder. Name of output file for a motif is \texttt{\inangles{name of motif}.thr}. Output files are created in a specified folder.
Each file consists of lines in the following format: 

\texttt{\inangles{threshold} \textit{tab} \inangles{corresponding \pvalue}}

Lines are sorted with thresholds ascending (thus \pvalue\ descending).

It takes about half a minute to preprocess the whole collection of $\sim$400 PWMs with default 
parameters using 1.5 GHz CPU.



\section{MACRO-APE: Matrix Comparison by Approximate P-value Estimation}
Let us have two PWMs with given threshold levels. The similarity between PWMs is
related to the number of words recognized by both PWMs
(or the aggregated probability of the word set under the given i.i.d. model).
To calculate this value we use generalized approach
described in [Touzet2007] for two PWMs simultaneously in a way similar to that in [Pape2008].
The number of words recognized by both PWMs can be used to construct a variant of Jaccard
similarity measure for motifs considered as sets of allowed words scoring no less than predefined thresholds.

Typical methods of PWM comparison are based on direct evaluation of matrix elements, for instance
by comparing matrices column by column (where different columns correspond to different 
positions of a transcription factor binding site).
On the other hand, in applications PWM is used as TFBS model
to identify “binding sites” by scanning a given sequence and identifying words with scores
no less than a threshold.

Thus, in reality a TFBS model is related to the set of words
scoring no less than the given threshold for the given PWM. It is desirable to construct a 
similarity measure for TFBS models based on the similarity between word sets recognized
by the matrices with given thresholds, rather than on  similarity between matrices per
se. Moreover, comparison-by-elements strategy requires the matrices to have algebraically 
comparable values (either frequencies or specifically scaled weights) which is not necessary if 
sets of recognized TFBS are compared.

MACRO-APE computes a similarity measure which directly accounts for similarity of
recognized word sets. This measure does not require PWM elements to be algebraically 
comparable and so it can be used to compare weight matrices constructed by different 
normalization / conversion strategies (e.g. log-odds with different pseudocounts and/or background normalization).

Let us have a position weight matrix of length $w$. The whole set of ACGT-alphabet 
words of length $w$ will be called the dictionary of size $N=4^w$. For a fixed threshold level $t$ one 
can calculate the fraction of the dictionary (i.e. the number of words $n$) scoring no less than the 
threshold. We will call the value of $n / N$ as the motif \pvalue.

Suppose we have two PWMs $m_1$, $m_2$ of length $w$ and some \pvalue\ levels $p_1$, $p_2$. For $m_1$
and $m_2$ we can estimate the thresholds $t_1$ and $t_2$ corresponding to $p_1$, $p_2$. Having PWMs with the 
corresponding thresholds we can estimate the fraction f of the dictionary recognized by both 
models, i.e. the size of the set of words scoring no less than $t_1$ on $m_1$ and no less than $t_2$ on $m_2$.

Moreover one can construct the Jaccard index 
$$J = \frac{\mathbf{A}\cap \mathbf{B}}{\mathbf{A}\cup \mathbf{B}}\,, $$
where $\mathbf{A}$ and $\mathbf{B}$ are 
sets of words recognized by $m_1$ and $m_2$ with the thresholds $t_1$ and $t_2$. If necessary one also can 
construct a Jaccard distance as $$d(\mathbf{A},\mathbf{B}) = 1 - J\,.$$

In the general case we have two PWMs of different widths, unknown optimal mutual 
alignment and orientation. For each possible alignment shift and orientation the matrices can be 
extended to the same length by adding zero-columns (not affecting either score or threshold) 
and then compared as the two models of the same width. Then one can determine the optimal 
shift and orientation by selecting the case with the highest Jaccard similarity. More formal and 
detailed explanation can be found in the corresponding macroape paper [MACROAPE].

\textbf{NOTE!} The reverse complementary transformation can be necessary to optimally align a given pair of matrices,
thus the background nucleotide composition for matrix comparison tools should be symmetrical, i.e. p(A) = p(T) and p(C) = p(G).


\subsection{EvalSimilarity}
\cmdname{EvalSimilarity} computes the similarity of two given motifs defined as a Jaccard similarity of sets of words recognized by each motif.
Optimal mutual alignment of the motifs is also estimated. Sets of recognized words are given by a PWM accompanied with threshold or a \pvalue. 

By default a set of recognized words is defined as top $0.05\%$ of words (i.e. \pvalue\ level of $0.0005$) ranked by a PWM.
It's possible to set required \pvalue\ with \cmdoption{-p \requiredarg{\pvalue}} option or to specify thresholds explicitly so that word sets contain all words passing corresponding thresholds. It can be accomplished using \cmdoption{--first-threshold \requiredarg{threshold}} and \cmdoption{--second-threshold \requiredarg{threshold}}.

In order to get intuition of Jaccard similarity scale and to better catch our output format, try these examples and take a look at corresponding motif logos (see the sample data):

\emph{TODO: here aligned motif logos should go}

\exampleof{rather similar motifs \texttt{KLF4\_f2} and \texttt{SP1\_f1}}
\EvalSimilarity{motifs/KLF4\_f2.pat\\ motifs/SP1\_f1.pat}
% \outputheader
% \cmdoutputfromfile{EvalSimilarityOutput(KLF4,SP1).txt}

\exampleof{the same motif \texttt{SP1\_f1} in opposite orientations}
\EvalSimilarity{motifs/SP1\_f1\_revcomp.pat\\ motifs/SP1\_f1.pat}
% \outputheader
% \cmdoutputfromfile{EvalSimilarityOutput(SP1,revcompSP1).txt}

\exampleof{significantly different motifs \texttt{SP1\_f1} and \texttt{GABPA\_f1}}
\EvalSimilarity{motifs/SP1\_f1.pat\\ motifs/GABPA\_f1.pat}
% \outputheader
% \cmdoutputfromfile{EvalSimilarityOutput(SP1,GABPA).txt}


By default \cmdname{EvalSimilarity} tests all possible mutual motif alignments in both orientations. 
A special option \cmdoption{--position} will force evaluating similarity with the explicitly specified motif alignment:

\cmdoption{--position \requiredarg{shift},\requiredarg{direct|revcomp}}

Option parameters are comma-separated, spaces not allowed; the position is defined for the second motif relative to the first.

Try the following examples:
\exampleof{rather similar motifs \texttt{KLF4\_f2} and \texttt{SP1\_f1} at optimal alignment}
\EvalSimilarity{motifs/KLF4\_f2.pat\\ motifs/SP1\_f1.pat\\ --position -1,direct}
% \outputheader
% \cmdoutputfromfile{EvalAlignmentOutput(RightPosition).txt}

\exampleof{rather similar motifs \texttt{KLF4\_f2} and \texttt{SP1\_f1} at completely wrong alignment}
\EvalSimilarity{motifs/KLF4\_f2.pat\\ motifs/SP1\_f1.pat\\ --position 3,revcomp}
% \outputheader
% \cmdoutputfromfile{EvalAlignmentOutput(WrongPosition).txt}

{\small
\textbf{Note!} By default \cmdname{EvalSimilarity} selects the thresholds corresponding to the \pvalue\ not 
less than requested (upper boundary) possibly making compared word sets larger (not to miss words with scores too close to the threshold).
This differs from \cmdname{FindThreshold} approach which, by default, uses 
lower boundary for \pvalue thus controlling the prediction rate more strictly.

It is very important to select upper \pvalue\ boundary for short PWMs. In case of given 
low \pvalues\ they can recognize no words at all (so the Jaccard measure may have zero 
numerator and zero denominator). For reasonable threshold levels both upper and lower 
boundaries usually produce very close similarity values, see the MACRO-APE paper for details [MACROAPE].

Nevertheless, one can override this behavior with \cmdoption{‘--boundary lower’} option. In such a case if 
any of supplied PWMs recognizes no words for a selected \pvalue, then similarity can not be 
correctly determined and macroape will report the similarity value of $-1$. 
}

\subsubsection{ScanCollection}
This tool uses a collection of motifs to find PWMs similar to a given query. The tool returns 
a list of all PWMs in the collection with corresponding similarity levels comparing to the query 
PWM. List is sorted by similarity in descending order so the PWMs similar to the query are 
located at the top of the list.
\emph{FIX: Actually not yet sorted}


NOTE! The shift and orientation are reported for PWMs from the collection relative to the query 
PWM.

\example{(search for motifs similar to \texttt{KLF4\_f2}, HOCOMOCO collection with uniform background), should take $\sim$1-2 minutes}
\ScanCollection{motifs/KLF4\_f2.pat\\ ./hocomoco\_ad\_uniform/}
% \outputheader[Output (STDOUT)]
% \cmdoutputfromfile{ScanCollectionOutput(KLF4,default).txt}

\emph{FIX: \cmdoption{--precise} has required parameter, not optional, yet}

One can also use the two-pass search mode when the top of the list is additionally 
reprocessed using a more precise discretization level. Second pass is executed only 
if \cmdoption{--precise [min\_similarity=0.01]}
key is specified. The precise search will recheck only the PWMs 
similar to the query with a similarity no less than \cmdoption{min\_similarity}. The results of the second pass 
will be marked by asterisk(*) in the output.

One can specify similarity cutoff with option \cmdoption{-c \requiredarg{similarity cutoff}} to discard 
comparison results with the resulting similarity less than a given value. By default results with similarity less than 0.05 are discarded.
In order to print comparison results for all PWMs in collection, one should specify \cmdoption{--all} option.

\example{(search PWMs similar to \texttt{KLF4\_f2}, uniform normalization, extended precision for the most similar PWMs), should take $\sim$10 seconds for 1.5 GHz CPU}
\ScanCollection{motifs/KLF4\_f2.pat\\ ./hocomoco\_ad\_uniform/\\ --precise}
% \outputheader[Output (STDOUT)]
% \cmdoutputfromfile{ScanCollectionOutput(KLF4,precise).txt}

To find similar PWMs using a particular \pvalue\ level one should use the \cmdoption{"-p"}~option.

\exampleof{should take $\sim$7 seconds for 1.5GHz CPU}
\ScanCollection{motifs/KLF4\_f2.pat\\ ./hocomoco\_ad\_uniform/\\ -p 0.001\\ -c 0.06 --precise 0.1}
% \outputheader[Output (STDOUT)]
% \cmdoutputfromfile{ScanCollectionOutput(KLF4,CustomPvalue,CustomFilters).txt}



\section{PERFECTOS-APE: Predicting Regulatory Functional Effect of SNPs by Approximate P-value Estimation.}
Variations in genome sequences are quite common. One widespread type of variations is represented by single nucleotide substitutions called single nucleotide variants~(SNVs) or, for a given population, single nucleotide polymorphisms~(SNPs).

SNVs in gene regulatory regions may affect gene expression through alterations in transcription factor binding sites.

PWM of transcription factor binding sites provides a score for any putative TFBS.
This score roughly represents binding affinity, thus allowing to estimate 
the impact of a given substitution through change in a score value.

As discussed earlier (section~\ref{ape-introduction}) scores are not directly comparable and do not have a unified scale. More convenient measure is the \pvalue\ - the probability to find a high-scoring word at random.

PERFECTOS-APE computes motif \pvalues\ for each sequence variant and calculates \pvalue\ fold change of a given substitution. Detailed algorithm for evaluating a fold change for a given TF and a substituion:

\begin{itemize}
\item Calculate PWM scores for putative TFBS overlapping a sequence variant.
\item Choose the best position and score for both sequence variants independently.
\item Estimate \pvalues\ for the best scores.
\item Compute fold change as the rate of \pvalues.
\end{itemize}

PERFECTOS-APE tests given SNVs against a whole collection of PWMs and yields (SNV, TF) pairs of SNVs that may significantly affect TF affinity.


\cmdname{SNPScan} takes a list of SNVs with flanking sequences and a motif collection and returns a list of predicted TFBS which were possibly disrupted by or emerged after a certain SNV.
If flanking sequence is too short it's padded with poly-N tail up to necessary length.
\usageheader
\SNPScan{\requiredarg{path to the collection of motifs} \requiredarg{path to the file with the list of SNVs} \optionalarg{options}}


\cmdname{SNPScan} has two filters. The first discards (SNV,~TF) pairs without TFBS prediction at any of nucleotide variants. \cmdname{SNPScan} treat a word as a putative TFBS if \pvalue\ of this word's score is not greater than the predefined threshold (0.0005 by default, changed via \cmdoption{--pvalue-cutoff} option:\\
\cmdname{--pvalue-cutoff \requiredarg{maximal \pvalue\ to be considered}}\\
or in short form:\\
\cmdname{-P \requiredarg{maximal \pvalue\ to be considered}}.

The second filter requires check \pvalue\ fold change to be large enough. By default fold change threshold is equal to $5$. It means that only SNVs causing \pvalue\ change of 5x and more ($FoldChange \ge 5$ or $FoldChange \le 1/5$) will be included in results. Fold change threshold can be specified using \cmdoption{--fold-change-cutoff}:\\
\cmdoption{--fold-change-cutoff \requiredarg{minimal fold change to be considered}}\\
or in short form:\\
\cmdoption{-F \requiredarg{minimal fold change to be considered}}

\cmdoption{--log-fold-change} option changes fold change from $\frac{\pvalue{}_1}{\pvalue{}_2}$ into
$\log_{2}{\frac{\pvalue{}_1}{\pvalue{}_2}}$ both in command-line parameter settings and output.

Option \cmdoption{--expand-region \requiredarg{length}} allows PWM hits to be located nearby but not strictly overlap the position with the nucleotide substitution.

When this option is specified, the PWM occurrence can be located up to \texttt{length} bp away from the SNV position.

This option is intended for analysis involving control data with SNVs not necessarily overlapping the binding sites.


The last but the most useful option is \cmdoption{--precalc} which forces \cmdname{SNPScan} to work with precalculated \pvalue,thresholds pairs performing binary search to evaluate the \pvalue\ instead of calculating motif score distribution each time from scratch. It can reduce total computation time in hundreds of times for large datasets.
As an input it requires a folder with precalculated (\pvalue,threshold) pairs - one for each motif:\\
\cmdoption{--precalc \requiredarg{path to a folder with precalculated \pvalue, threshold pairs}}

These precalculated score distributions are to be produced by a \cmdname{PreprocessCollection} from APE toolbox.
Please refer to the respective section for details.


\example{}
\SNPScan{./hocomoco/pwms/ snp.txt --precalc ./collection\_thresholds}
\SNPScan{./hocomoco/pcms/ snp.txt --pcm --discretization 10 --background 0.2,0.3,0.3,0.2}

\subsubsection{Output data format}

\cmdname{SNPScan} prints all results to standard output, errors and messages go into standard error stream. First line of output is a header of table. Latter lines are rows of this table. Columns are:
\begin{itemize}
\item Name of sequence containing SNV
\item TF motif name
\item for the first allele variant:
\begin{itemize}
\item   the best position and strand of putative TF-DNA binding
\item   nucleotide word corresponding to the best binding sequence among all other words in sequence, intersecting SNV
\end{itemize}
\item the same two columns for the second allele variant
\item allele variants
\item \pvalue\ for the first allele variant
\item \pvalue\ for the second allele variant
\item fold change (the first \pvalue\ divided by the second \pvalue)
\end{itemize}

Position of the best binding place is given for the leftmost boundary of a binding sequence (independent of strand orientation). The SNV location is at zero, so the TFBS coordinates are always less or equal to zero. Strand is denoted as `direct` or `revcomp`. Words are given at the relevant strand (i.e. reverse-complement transformation is applied if necessary).

More compact output format can be produced using the \cmdoption{--compact} option.

The resulting table will have the following columns:
\begin{itemize}
\item Name of sequence containing SNV
\item TF motif name
\item \pvalue\ for the first allele variant
\item \pvalue\ for the second allele variant
\item the best position and strand of putative TF-DNA binding for the first allele variant
\item the best position and strand of putative TF-DNA binding for the second allele variant
\end{itemize}

Please note that fold change and word sequences are not shown (comparing to the default output).
Strand information is given as +/- form (versus direct/revcomp in the default output).
\pvalues are rounded up to three significant digits.

This option is intended to process huge lists of SNVs and reduce the output (~2.5x less size).



\section{Data formats}
% Augment with information about transposed matrices with a link to options
\subsection{Position matrix format description}
All tools described below use the following matrix file format (each binding site position 
corresponds to a separate line):

\texttt{\begin{tabular}{llll}
some\_header\\
pos1\_A\_weight & pos1\_C\_weight & pos1\_G\_weight & pos1\_T\_weight\\
\ldots\\
posw\_A\_weight & posw\_C\_weight & posw\_G\_weight & posw\_T\_weight
\end{tabular}
}

Position matrix format is appliable for all kinds of positional matrices: positional weight(PWM), count(PCM) and probability/frequency(PPM). Matrix elements meaning changes accordingly.

The number of the lines corresponds to the PWM width. If given, header will be treated as a motif name, otherwise filename will stand for motif name. Header has optional \texttt{"\textgreater"} sign at line start.

If necessary it's possible to read matrices, with nucleotides in rows and positions in columns. In order to do this, specify \cmdoption{--transpose} option.

\exampleof{PWM similar to HOCOMOCO transcription factor motif for KLF4}
{\small\verbatiminput{./MotifSamples/KLF4_f2_alike.pwm}}

More real-life examples are provided with the package.

Dinucleotide version of tools work with dinucleotide motifs. Dinucleotide positional matrices have similar format but contain 16 columns instead of 4. Columns go in order: AA, AC, AG, AT, CA, CC, \dots, TT. It's also possible to use mononucleotide motifs in dinucleotide tools. For rationales and details take a look at \cmdoption{--from-mono} option.

\cmdname{SNPScan} uses a list of sequences with SNVs as input data.

The list of sequences with SNVs should be given in a single plain text file. 
Each sequence should be presented at a separate line using the following format:\\
\texttt{\inangles{SNV name} \inangles{left flank}[\inangles{variant 1}/\inangles{variant 2}]\inangles{right flank}}

SNV name shouldn't contain empty delimiters (spaces or tabs). 
Sequence consists of two allele variants in square brackets, separated with `/`, and flanking sequences at both sides. Length of flanking sequences should be sufficient to place the longest motif of a given collection (so it is advised to provide 25-30bp at each side) into all positions relative to a nucleotide substitution position. 

So, first two columns are SNV name and SNV sequence. Later columns (if present) are ignored, thus can contain any data.

\exampleof{SNV list}
\noindent\texttt{%
\# Text after "\#" doesn't matter\\
\# It's possible to include any number of comment lines into input\\
rs10040172 gattgcagttactga[G/A]tggtacagacatcgt\quad Anything\\
rs10116271 gtggggaagaggtct[C/T]gtagaggcgatgatt\quad can go\\
rs10208293 ttatgtccagtacct[A/G]tggaccctccttgtg\quad after first\\
rs10431961 ggtcaggcggcgtcg[C/T]cggtacgctctgagc\quad two columns
}

\vspace{1.5ex}
Note that lines starting with \texttt{\#} are considered as comments and thus ignored by \cmdname{SNPScan}.



\section{Additional command-line options}
Many additional options are available for *-APE tools. The options should be provided after the required arguments. There are common options among all *-APE tools as well as tool-specific options (already described in the respective sections).

This section covers common options: those altering input data format and those affecting calculation parameters.
The first class of options allows using input motifs as different matrices (counts, PCM or probabilities, PPM) instead of default weights (PWM), load matrices in transposed format and use mononucleotide motifs in dinucleotide tools.
The second class of options allows to set the background model, select \pvalue\ evaluation mode, limit memory consumption and so on.

For a full list of options for a particular tool please run the tool with the \cmdoption{--help} command line option.


\subsubsection{Dinucleotide motifs format}
Dinucleotide version of tools work with dinucleotide motifs. Dinucleotide positional matrices have similar format but contain 16 columns instead of 4. Columns go in order: AA, AC, AG, AT, CA, CC, \dots, TT.

It is also possible to use mononucleotide PWMs instead of dinucleotide PWMs. Conversion PWM~$\rightarrow$~DiPWM will be done internally in such a way that each word has the same score on DiPWM as it had on PWM. It's done with \cmdoption{--from-mono} option. 

\cmdoption{--first-from-mono} and family (analogous to options described in \ref{TransposeOption} section) are also available.

Obtaining dinucleotide motifs from mononucleotide can help you to compare mononucleotide PWM with dinucleotide PWM. Another reason to use them - is a study of mononucleotide motif properties on dinucleotide background.

{\small\textbf{Notice:} scores of words on discreted PWM and corresponding diPWM can be slightly different due to a discretization step performed after PWM~$\rightarrow$~DiPWM conversion. This discrepancy shouldn't worry you, it's small enough and goes to zero with discretization increase.}


% Here we should insert some link to pcm-pwm conversions
\subsubsection{\cmdoption{--transpose} option for motifs specified in different orientation}\label{TransposeOption}
You can use motifs with nucleotides in rows, by specifying \cmdoption{--transpose}. The only difference in format is matrix orientation, header remains the same. This option is available for each tool.

Macroape tools work with pairs of motifs and \cmdoption{--transpose} option mean that both motif sources are transposed. In tool \texttt{macroape.EvalSimilarity} one can use \cmdoption{--first-transpose} or \cmdoption{--second-transpose} instead to use one matrix specified in horizontal orientation, another in vertical. Similar options for \texttt{macroape.ScanCollection} are named \cmdoption{--collection-transpose} and \cmdoption{--query-transpose}.


Nucleotide frequencies of a background model can be specified in optional arguments, e.g. \cmdoption{--background} or \cmdoption{--query-background}. All background options use the same format with a single required argument: \cmdoption{--background \requiredarg{value}}.

Default background model is a \texttt{wordwise} model. It means that all our calculations assume uniform nucleotide distribution and the exact number of words is used everywhere instead of probabilities of a word set.
E.g. \texttt{FindPvalue} will calculate not the probability of a random word score to pass the threshold but a fraction of words scoring greater than threshold estimating the exact number of such words.

A number of words is a more natural and intuitive to use, especially if the background model cannot be properly selected
thus we suggest "wordwise" mode by default.

Wordwise mode can be specified explicitly, e.g. using \cmdoption{--background wordwise} key.

All following formats are different ways to specify frequencies of each nucleotide:
\begin{itemize}
\item The most simple nucleotide background model is uniform, each nucleotide has the same probability to occur. Option format is: \cmdoption{--background uniform}. This is close to wordwise mode, but word set probabilities are used and reported instead of raw counts of words.
\item It is also possible to specify a fixed GC-content~(in range $0$ to $1$): \cmdoption{--background \requiredarg{GC-content}}. E.g.~\cmdoption{"--background~0.6"}
\item The most detailed format is to explicitly specify nucleotide frequencies: \cmdoption{--background \requiredarg{$p_A,p_C,p_G,p_T$}}. E.g.~\cmdoption{"--background 0.2,0.3,0.3,0.2"} will define the same frequencies as for GC-content of $0.6$. Note that nucleotide frequencies should be given in alphabetical ACGT-order separated with commas. 
\end{itemize}

\textbf{Note:} No spaces between frequencies are allowed (commas only). Sum of frequencies should be equal to $1.0$.

\subsubsection{Dinucleotide background}
Dinucleotide background has the same options: wordwise, uniform, GC-content and dinucleotide frequencies. Worwise, uniform and GC-content backgrounds are effectively the same as mononucleotide ones and don't carry nucleotide interdependencies.

Dinucleotide frequencies in turn require some additional notes. Dinucleotide frequencies go in standard dinucleotide order: AA, AC, AG, \dots, TT~--- 16 terms. Each value means probability of a specific dinucleotide. These probabilities are \textbf{not} conditional probabilities used by an algorithm, conditional probabilities are calculated internally. Be careful if you got used to use Markov model background. Again, list of frequencies is comma-separated, no spaces allowed, sum of probabilities should be equal to 1.0.

Also one can specify mononucleotide ACGT-frequencies background even in dinucleotide tools. It will be recognized automatically when 4 values are specified instead of 16 ones. In such a case dinucleotide background will treat nucleotide as non-depending from previous one.


\subsection{Additional command-line options}

\subsubsection{Specifying custom discretization level}
For a more precise result \cmdoption{--discretization \requiredarg{discretization rate}} or \cmdoption{-d \requiredarg{discretization rate}} command line key can be used to explicitly set
the discretization level for PWM elements, like \cmdoption{"--discretization 100000"} (see the "PWM discretization" subsection in the "Formal math" section below).
The discretization level of $10^5$ corresponds to the precision of PWM elements up to 5 decimal places.
A larger number of decimal places results in increased precision and computational time.
The default setting of $10^4$ for single-motif tools and $10^1$ for motif comparison tools gives reasonable "time-precision" tradeoff.

\subsubsection{Specifying custom \pvalue\ level}
All tools in \texttt{MACRO-APE} package estimate motif threshold by a \pvalue\ for further use. By default \pvalue\ level of 0.0005 is assumed.
It can be overriden with \cmdoption{--pvalue \requiredarg{\pvalue}} or \cmdoption{-p \requiredarg{\pvalue}} option key.

\subsubsection{Choose proper threshold by a \pvalue}
All *-APE tools except \cmdname{ape.FindPvalue} and \cmdname{perfectosape.SNPScan} perform internal \pvalue\ to threshold conversion.
Since PWM \pvalues\ have discrete distribution a given \pvalue\ can be achieved only approximately.
A fixed threshold corresponds to the actual \pvalue\ which is smaller or larger than the requested \pvalue.

The boundary selection can be done using \cmdoption{--boundary \requiredarg{lower|upper}}.

For model comparison by default we use the upper boundary for the \pvalue~(so even at low given \pvalues\ PWMs recognize some words and thus the models can be 
compared). If searching for a threshold corresponding to the given \pvalue\ we report the lower boundary of the \pvalue\ by default (to properly control
the positive prediction rate corresponding to a given threshold).

\textbf{Note:} \texttt{lower} boundary means that \pvalue\ will be not greater than the requested one.
The threshold for \texttt{lower} P-value will be greater than the threshold for \texttt{upper} boundary \pvalue.


\subsubsection{Limiting CPU and memory consumption}
It's possible to create an artificially arranged PWM whose score distribution
 will grow exponentially with length and thus can take a lot of memory and time for computation.
 This option is mostly designed to prevent *-APE tools from unnormal CPU and memory consumption.
 If hash size exceeded a given limit, tools cancel calculations with \texttt{"Hash overflow"} error message.
 In such case user can manually expand hash size limits or lower discretization level.
  \begin{itemize}
  \item \cmdoption{--max-hash-size \requiredarg{size}}: set the internal hash (used for score distribution calculation) size limit. Default value is $10^7$
  \item \cmdoption{--max-2d-hash-size \requiredarg{size}}: set the internal two-dimensional hash size limit (used for PWM comparison in \texttt{MACRO-APE} toolbox). Default value is $10^4$.
  \end{itemize}



\section{Formal math}
  \subsection{PWM discretization}
Following the general idea described in [Touzet2007] we can effectively calculate the \pvalue\ 
for a given PWM with a fixed precision and a given threshold value. The algorithm of Touzet \textit{et al.}
efficiently processes matrices with integer elements. The matrices with real values are transformed
into integer value matrices by multiplying each value by discretization constant and truncating the decimals.

Effectively this is similar to rounding up real values leaving only the fixed number of
decimal places. The higher discretization level will result in a more accurate \pvalue\ calculation 
and an increased computational time. 

Please note, that in contrast to the original Touzet algorithm here we applying "ceil" 
operation to the matrix elements (instead of "floor" in the original paper of Touzet). This allows 
us to have a strict upper boundary of the threshold for a given \pvalue.

We use the default discretization level of $10^4$ to perform calculations with accuracy up to
four significant digits for single-PWM tools from APE toolbox.

For motif comparison the straightforward discretization by rounding up to
the nearest integer is used by default for a fast and rough search through the motif collection. 
The default level of $10$ (one decimal place) is used for a more precise search of similar motifs.

Thus in our case discretization is the transformation as follows: discretized~$S$ is
$S$~multiplied by discretization level~$V$ and rounded up to the nearest integer value.

\texttt{
\begin{tabular}{ll}
Example:\\
S = 1.6734\\
discretization V=1 & discretized S = $\lceil 1.6734 \rceil$ = 2\\
discretization V=10 & discretized S = $\lceil 16.734 \rceil$ = 17\\
discretization V=100 & discretized S = $\lceil 167.34 \rceil$ = 168
\end{tabular}
}

Discretization will generally preserve the word score ranking with the common exception 
for words that would obtain identical scores. The main advantage of the discretization 
is decreasing of the number of possible scores so the set of all possible scores can be 
enumerated more effectively.

  \subsubsection{Obtaining PWM from PCM and PPM models}
One can specify motifs in PCM or PPM formats. If so, additional step of PCM~$\rightarrow$~PWM or PPM~$\rightarrow$~PWM conversion to be performed. In such a case \cmdoption{--pcm} or \cmdoption{--ppm} should be specified.

Matrix of positional counts (PCM) can be transformed into PWM according to the formula used in [Lifanov2003]:
$$ PWM_{\alpha,j} = \ln\frac{ PCM_{\alpha,j} + aq_{\alpha} }{ (W+a)q_{\alpha} }\,, $$

where $\alpha$ is a nucleotide index and $j$ is a position index; $W$ is the total weight of the alignment (or the number of aligned words), $a$ is the 
pseudocount value selected by default as the $\ln(W)$, and $q_{\alpha}$ is the background probability of nucleotide letter $\alpha$.

This conversion procedure isn't the only one possible, but all our tools use this procedure. If you need another one, you can manually convert PCMs into PWMs according to your owm algorithm.

This algorithm has several parameters. At first, pseudocount $a$ which has a default value of $\ln(W)$ can be redefined. In order to set pseudocount to a constant, one can use \cmdoption{"--pseudocount \requiredarg{value}"} option.

PPM~$\rightarrow$~PWM conversion is done in two stages. First PPM is multiplied by a constant alignment weight $W$ to obtain a PCM. Then this PCM is converted to a PWM as described above. 
A user should supply alignment weight explicitly by the \cmdoption{--effective-count \requiredarg{alignment weight}} option. If this information is not given, alignment weight of 100.0 will be used as a default assumption.

Backgrounds specified for tool-specific purposes will be also used for PCM~$\rightarrow$~PWM conversion.

\emph{TODO: Dibackground motifs conversion strategy}


\section*{References}
[MACROAPE] Algorithms Mol Biol. 2013, 8:23.
Jaccard index based similarity measure to compare transcription factor binding site models.
Ilya E Vorontsov, Ivan V Kulakovskiy and Vsevolod J Makeev

[Touzet2007] Algorithms Mol Biol. 2007 Dec 11;2:15. Efficient and accurate \pvalue  
computation for Position Weight Matrices. Touzet H, Varré JS.

[Pape2008] Bioinformatics. 2008 Feb 1;24(3):350-7. Epub 2008 Jan 2. Natural similarity 
measures between position frequency matrices with an application to clustering. Pape UJ, 
Rahmann S, Vingron M.

[Lifanov2003] Genome Res. 2003 Apr;13(4):579-88. Homotypic regulatory clusters in 
Drosophila. Lifanov AP, Makeev VJ, Nazina. AG, Papatsenko DA.
\end{document}
