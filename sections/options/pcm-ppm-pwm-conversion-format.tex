\subsubsection{Obtaining PWM from PCM and PPM models}
One can specify motifs in PCM or PPM formats. If so, additional step of PCM~$\rightarrow$~PWM or PPM~$\rightarrow$~PWM conversion to be performed. In such a case \cmdoption{--pcm} or \cmdoption{--ppm} should be specified.

Matrix of positional counts (PCM) can be transformed into PWM according to the formula used in [Lifanov2003]:
$$ PWM_{\alpha,j} = \ln\frac{ PCM_{\alpha,j} + aq_{\alpha} }{ (W+a)q_{\alpha} }\,, $$

where $\alpha$ is a nucleotide index and $j$ is a position index; $W$ is the total weight of the alignment (or the number of aligned words), $a$ is the 
pseudocount value selected by default as the $\ln(W)$, and $q_{\alpha}$ is the background probability of nucleotide letter $\alpha$.

This conversion procedure isn't the only one possible, but all our tools use this procedure. If you need another one, you can manually convert PCMs into PWMs according to your owm algorithm.

This algorithm has several parameters. At first, pseudocount $a$ which has a default value of $\ln(W)$ can be redefined. In order to set pseudocount to a constant, one can use \cmdoption{"--pseudocount \requiredarg{value}"} option.

PPM~$\rightarrow$~PWM conversion is done in two stages. First PPM is multiplied by a constant alignment weight $W$ to obtain a PCM. Then this PCM is converted to a PWM as described above. 
A user should supply alignment weight explicitly by the \cmdoption{--effective-count \requiredarg{alignment weight}} option. If this information is not given, alignment weight of 100.0 will be used as a default assumption.

Backgrounds specified for tool-specific purposes will be also used for PCM~$\rightarrow$~PWM conversion.

\emph{TODO: Dibackground motifs conversion strategy}
