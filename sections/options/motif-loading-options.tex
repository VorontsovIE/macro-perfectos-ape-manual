\subsection{Motif loading options}
By default motifs are expected to be provided as position weight matrices in a nucleotides-in-columns plain text format. Basic tools use mononucleotide positional matrices, dinucleotide tools use dinucleotide matrices. However, many motif collections provide position frequency matrices (PFMs, or probability matrices, PPMs) or position count matrices (PCMs). *-APE tools can convert these matrices to PWMs internally (using a log-odds-like transformation as in [Lifanov2003], see the "Formal math" section below).


\begin{figure}[h]
\centering
$$\begin{array}{ccccc}
PPM     &  \stackrel{\cmdoption{--ppm}}{\longrightarrow}  & PWM \bullet               & \stackrel{\cmdoption{--pcm}}{\longleftarrow}  &  PCM  \\
        &                                                 &          |                &                                               &  \\
        &                                                 & \cmdoption{--from-mono}   &                                               &  \\
        &                                                 &     \downarrow            &                                               &  \\
DiPPM   &  \stackrel{\cmdoption{--ppm}}{\longrightarrow}  & DiPWM \bullet             & \stackrel{\cmdoption{--pcm}}{\longleftarrow}  &  DiPCM
\end{array}$$
\caption{Command-line options to read a motif from non-PWM motif models. Conversion end-points are marked with bullets.}\label{motif-conversion-types}
\end{figure}


\begin{figure}[h]
\centering
$$\begin{array}{ccccc}
PPM   & \stackrel{\cmdoption{--effective-count}}{\longrightarrow} & PCM   & \stackrel{\cmdoption{--background}}{\stackrel{\cmdoption{--pseudocount}}{\longrightarrow}} & PWM \bullet \\
      &                                                           &       &                                                                                            & \downarrow \\
DiPPM & \stackrel{\cmdoption{--effective-count}}{\longrightarrow} & DiPCM & \stackrel{\stackrel{\cmdoption{--background}}{\cmdoption{--pseudocount}}}{\longrightarrow} & DiPWM \bullet
\end{array}$$
\caption{Motif transformations configuration options. Conversion end-points are marked with bullets.}\label{motif-conversion-configuration}
\end{figure}


\subsubsection{Obtaining PWM from PCM and PPM models}
For loading motif from position count matrices one should specify \cmdoption{--pcm} option. For loading motifs from position frequency matrices one should specify \cmdoption{--ppm} option. (see fig.~\ref{motif-conversion-types})

In such cases, additional step of \PcmToPwm\ or \PpmToPwm\ conversion would be performed.

These data model transformations can be configured.

\PcmToPwm\ conversion is described in a [PCM to PWM math] section. It's possible to manually specify a fixed psedocount $a$ with \cmdoption{--pseudocount \requiredarg{a}} option. When not specified, pseudocount is derived from alignment weight $W$: $a =\ln(W)$.

\PpmToPwm\ conversion is done in two stages. At first PPM is multiplied by a constant alignment weight $W$ to obtain a PCM. Then this PCM is converted to a PWM as described above.
A user should supply alignment weight $W$ explicitly by the \cmdoption{--effective-count \requiredarg{W}} option. If this information is not given, alignment weight of 100.0 will be used as a default assumption.

\PcmToPwm\ conversion will also take a background specified for tool-specific purposes into account.

DiPCMs are converted to DiPWMs using the same formula as for \PcmToPwm\ conversion, the only difference is that now nucleotide index goes through 16 dinucleotides at each position instead of 4 nucleotides.

All configuration options can be inspected on a fig.~\ref{motif-conversion-configuration}.

\subsubsection{Obtaining dinucleotide motifs from mononucleotide ones}
Dinucleotide *-APE tools take dinucleotide motifs as input parameters. But there is an option \cmdoption{--from-mono} which allows to use basic mononucleotide motifs instead so that \PwmToDiPwm\ will be done internally. It can be useful in following cases:
\begin{itemize}
\item Comparison of dinucleotide motif against mononucleotide one. In this case one motif should be loaded as dinucleotide motif, the rest - as mononucleotide motif internally converted to a dinucleotide motif. Further comparison performs on two dinucleotide motifs.
\item Study of mononucleotide motif properties on dinucleotide background. It isn't possible to specify dinucleotide background for a mononucleotide tool, but is possible to specify mononucleotide motif and dinucleotide background for a dinucleotide tool. 
\end{itemize}

\PwmToDiPwm\ is done in such a way that each word has the same score on diPWM as it had on PWM.

{\small\textbf{Notice:} scores of words on discreted PWM and corresponding diPWM can be slightly different due to a discretization step performed after \PwmToDiPwm\ conversion. This discrepancy shouldn't worry you, it's small enough and goes to zero with discretization increase.}


When both \cmdoption{--pcm} and \cmdoption{--from-mono} options are specified, the conversion is done in two stages (see fig.~\ref{motif-conversion-configuration}). First, \PcmToPwm\ transformation is applied and then \PwmToDiPwm\ transformation is applied. 
Background model used in \PcmToPwm\ conversion should be given as mononucleotide letter frequencies ("Bernoulli" i.i.d. random model). Background provided to a dinucleotide tool should be given as dinucleotide frequencies. In this case mononucleotide frequencies are estimated by averaging dinucleotide background:
$$p_{\alpha} = \frac{\sum_{\beta}p_{\alpha\beta} + \sum_{\beta}p_{\beta\alpha}}{2}$$


\subsubsection{\cmdoption{--transpose} option}\label{TransposeOption}
One can load motifs from nucleotides-in-rows using \cmdoption{--transpose} option. The only difference in format is matrix orientation, header remains the same~(see "Position matrix format" section).
