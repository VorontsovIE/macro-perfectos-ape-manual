\subsubsection{Obtaining PWM from PCM and PPM models}
For loading motif from position count matrices one should specify \cmdoption{--pcm} option. For loading motifs from position frequency matrices one should specify \cmdoption{--ppm} option. (see fig.~\ref{motif-conversion-types})

In such cases, additional step of \PcmToPwm\ or \PpmToPwm\ conversion would be performed.

These data model transformations can be configured.

\PcmToPwm\ conversion is described in a [PCM to PWM math] section. It's possible to manually specify a fixed psedocount $a$ with \cmdoption{--pseudocount \requiredarg{a}} option. When not specified, pseudocount is derived from alignment weight $W$: $a =\ln(W)$.

\PpmToPwm\ conversion is done in two stages. At first PPM is multiplied by a constant alignment weight $W$ to obtain a PCM. Then this PCM is converted to a PWM as described above.
A user should supply alignment weight $W$ explicitly by the \cmdoption{--effective-count \requiredarg{W}} option. If this information is not given, alignment weight of 100.0 will be used as a default assumption.

\PcmToPwm\ conversion will also take a background specified for tool-specific purposes into account.

DiPCMs are converted to DiPWMs using the same formula as for \PcmToPwm\ conversion, the only difference is that now nucleotide index goes through 16 dinucleotides at each position instead of 4 nucleotides.

All configuration options can be inspected on a fig.~\ref{motif-conversion-configuration}.
