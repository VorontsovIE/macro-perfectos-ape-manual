\subsection{Option families}
{\small

Options are grouped into "families" of options with similar names but different prefixes. For example \cmdname{macroape.di.EvalSimilarity} tool, has an option \cmdoption{--from-mono}. This option creates dinucleotide motifs by loading mononucleotide matrices. In turn, \cmdoption{--first-from-mono} options forces loading of the first motif from mononucleotide input and \cmdoption{--second-from-mono} does the same for the second motif.

Similar options for \cmdname{macroape.ScanCollection} are named \cmdoption{--query-from-mono} and \cmdoption{--collection-from-mono}. Option \cmdoption{--query-from-mono} requires mononucleotide query matrix, and \cmdoption{--collection-from-mono} means that each motif in collection should be loaded from mononucleotide matrix. The same is appliable for \cmdoption{--background}.

Such triples of options are typically listed in the help string like this: \cmdoption{--[first-|second-]from-mono}. It means that one can use both prefixed and non-prefixed options. Possible prefixes are given in square brackets separated with a pipe sign \texttt{"|"}.

\textbf{Note:} Prefixed options exist only in a long form. E.g. one can use both \cmdoption{-b} and \cmdoption{--background} as synonymous but for there is no short analogue for \cmdoption{--first-background}.

\textbf{Note:} Presence of separate options for each of used motifs doesn't necessarily involve existence of a common option. E.g. \cmdname{macroape.EvalSimilarity} has \cmdoption{--first-threshold} and \cmdoption{--second-threshold} options but doesn't have \cmdoption{--threshold} since it generally makes no sense to use the same algebraical threshold value for two independent PWMs (common \pvalue\ level in turn is a reasonable parameter).
}
