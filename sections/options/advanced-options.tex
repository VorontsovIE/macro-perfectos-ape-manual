\section{Advanced command-line options}
\emph{TODO: Many of these options has been already described.}

Additional options for macroape tools:
\cmdoption{--help}: show information on command-line parameters
\cmdoption{-b \requiredarg{probA,probC,probG,probT}}: set the background nucleotide composition
\cmdoption{--max-hash-size \requiredarg{size}}: set the internal hash size limit
\cmdoption{--pcm}: provide motif matrixes as PCM instead of PWM, conversion to be done internally

By default the maximum hash size is limited by $10^7$. This can be not enough to process an 
extremely high discretization level (see below) or specifically arranged artificial matrices.

Additional \cmdoption{--max-2d-hash-size \requiredarg{size}} can be used for \cmdname{EvalSimilarity}, 
\cmdname{EvalAlignment} and \cmdname{ScanCollection} tools. It sets the internal two-dimensional hash size used 
for PWM comparison. By default the total size is limited by $10^4$.

The error message \texttt{"Hash overflow in PWM/PWMCompare"} means the maximum Hash 
size should be increased by \cmdoption{"--max-hash-size/--max-2d-hash-size"} key.

The \cmdoption{“-b”} key can be used to supply a given background nucleotide composition so 
the words in the dictionary became weighted according to corresponding probabilities to be 
generated by i.i.d. random model.

NOTE! The background nucleotide composition for searching similar motifs in a collection 
is stored within the collection yaml-file (so you should set the background during collection 
preprocessing via \cmdname{PrecalculateThresholds}).

NOTE! The reverse complementary transform can be necessary to optimally align matrices, 
thus the background nucleotide composition for matrix comparison (\cmdname{EvalSimilarity} and 
\cmdname{PrecalculateThresholds} tools) should be symmetrical, i.e. p(A) = p(T) and p(C) = p(G).

The order of nucleotides is assumed alphabetical, so the \cmdoption{“-b 0.2,0.3,0.3,0.2”} will 
correspond to p(A,C,G,T) = {0.2, 0.3, 0.3, 0.2}. In case of a given background model tools print 
out not the number of words but the probability to randomly draw a word scoring no less than 
the threshold from the complete dictionary of words weighted according to the given background 
model.

All tools except \cmdname{FindPvalue} have additional option:
\cmdoption{--boundary \requiredarg{lower|upper}}: prefer threshold yielding actual \pvalue\ below/above than 
the requested \pvalue.

Since PWM \pvalues\ have discrete distribution a given \pvalue\ can be achieved only 
approximately. For model comparison by default we use the lower boundary for the threshold 
(so even at low given \pvalues\ PWMs recognize some words and thus the models can be 
compares). If searching for a threshold corresponding to the given \pvalue\ we report the upper 
boundary of the threshold by default.

\cmdname{PrecalculateThresholds} and \cmdname{ScanCollection} have additional option \cmdoption{--silent} which 
disables the output of progress information (printed to stderr by default).

