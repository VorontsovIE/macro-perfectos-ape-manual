APE tools are designed to properly convert PWM thresholds to \pvalues\ and vice versa.

Position weight matrix~(PWM) of DNA motifs assigns a score to each "word" (nucleotide sequence of a fixed length $l$). It makes 
possible to range the words by their scores, e.g. corresponding to predicted transcription affinity for PWMs of transcription factor binding sites~(TFBS). Given a threshold, one can divide all $l$-mers into two subsets: words whose score are not less than the threshold and the rest. Typically, the words passing the score threshold are selected for downstream analysis, e.g. they are considered as putative transcription factor binding sites. 

What is important, the threshold values are not directly comparable for different PWMs. One strategy to have a unified scale is to use motif \pvalues\ instead.

The \pvalue\ of a certain PWM and a score threshold is the probability to generate a word with the score not less than the threshold at random.

Inverse task is to estimate a threshold for a predefined \pvalue. In particular this allows to select a PWM score threshold corresponding to a predefined positive prediction rate across the $l$-mer dictionary (e.g. only $x\%$ of words are predicted as putative TFBS).

Our tools perform threshold -- \pvalue\ conversion implementing a dynamic programming algorithm on a granulated (discretized) PWM models using a simplified approach comparing to that described in [Touzet].

More details on \pvalues, thresholds and the algorithm are provided in the MACRO-APE paper. [MACROAPE] \urldecorated{http://www.almob.org/content/8/1/23}
