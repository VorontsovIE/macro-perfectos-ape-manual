\emph{Text from old manual abstract. Propose it to be removed, but not sure about paper references. Should we leave Pape paper reference?}

\emph{Let us have two PWMs with given threshold levels. Using macroape software one can 
calculate the number of words recognized by both PWMs (or the aggregated probability of the 
word set under the given i.i.d. model). To calculate this value we use generalized approach 
described in [Touzet2007] for two PWMs simultaneously in a way similar to that in [Pape2008]
. The number of words recognized by both PWMs can be used to construct a variant of Jaccard 
similarity measure for motifs considered as sets of allowed words.}

Typical methods of comparing PWMs are based on direct comparison of elements, for instance 
by comparing matrices column by column (where different columns correspond to different 
positions of a binding site). On the other hand, in applications PWM is used as TFBS model 
to identify “binding sites” by scanning a given text and identifying words having PWM scores 
no less than a preselected threshold. So, in reality a TFBS model is related to the set of words 
scoring no less than the given threshold for the given PWM. It is desirable to construct a 
similarity measure for TFBS models according to the similarity between word sets recognized 
by the matrices with given thresholds, rather than according to similarity between matrices per 
se. Moreover, comparison-by-elements strategy requires the matrices to have algebraically 
comparable values (either frequencies or specifically scaled weights) which is not necessary if 
sets of recognized TFBS are compared.

Here we present a similarity measure which directly accounts for the similarity of 
recognized word sets. This measure does not require PWM elements to be algebraically 
comparable and so it can be used to compare weight matrices constructed by different 
normalization / conversion strategies.

Let us have a position weight matrix of length $w$. The whole set of ACGT-alphabet 
words of length $w$ will be called the dictionary of size $N=4^w$. For a fixed threshold level $t$ one 
can calculate the fraction of the dictionary (i.e. the number of words $n$) scoring no less than the 
threshold. We will call the value of $n / N$ as the motif \pvalue.

Suppose we have two PWMs $m_1$, $m_2$ of length $w$ and some \pvalue\ levels $p_1$, $p_2$. For $m_1$
and $m_2$ we can estimate the thresholds $t_1$ and $t_2$ corresponding to $p_1$, $p_2$. Having PWMs with the 
corresponding thresholds we can estimate the fraction f of the dictionary recognized by both 
models, i.e. the size of the set of words scoring no less than $t_1$ on $m_1$ and no less than $t_2$ on $m_2$.

Moreover one can construct the Jaccard index 
$$J = \frac{\mathbf{A}\cap \mathbf{B}}{\mathbf{A}\cup \mathbf{B}}\,, $$
where $\mathbf{A}$ and $\mathbf{B}$ are 
sets of words recognized by $m_1$ and $m_2$ with the thresholds $t_1$ and $t_2$. If necessary one also can 
construct a Jaccard distance as $$d(\mathbf{A},\mathbf{B}) = 1 - J\,.$$

In the general case we have two PWMs of different widths, unknown optimal mutual 
alignment and orientation. For each possible alignment shift and orientation the matrices can be 
extended to the same length by adding zero-columns (not affecting either score or threshold) 
and then compared as the two models of the same width. Then one can determine the optimal 
shift and orientation by selecting the case with the highest Jaccard similarity. More formal and 
detailed explanation can be found in the corresponding macroape paper [MACROAPE].

\emph{
I don't understand rationales of the next paragraph. It should probably be removed.}

\emph{The comparison strategy does not require PWMs to be directly comparable (so PWMs can be 
constructed from position count matrices, PCMs, using completely different approaches). Still it 
is prudent to use similarly processed matrices for collection and a query motif in motif scanning procedure. }
