\cmdname{SNPScan} takes a list of SNVs with flanking sequences and a motif collection and returns a list of predicted TFBS which were possibly disrupted by or emerged after a certain SNV.
\usageheader
\SNPScan{\requiredarg{path to the collection of motifs} \requiredarg{path to the file with the list of SNVs} \optionalarg{options}}


\cmdname{SNPScan} has two filters. The first discards (SNV,~TF) pairs without TFBS prediction at any of nucleotide variants. \cmdname{SNPScan} treat a word as a putative TFBS if \pvalue\ of this word's score is not greater than the predefined threshold (0.0005 by default, changed via \cmdoption{--pvalue-cutoff} option:\\
\cmdname{--pvalue-cutoff \requiredarg{maximal \pvalue\ to be considered}}\\
or in short form:\\
\cmdname{-P \requiredarg{maximal \pvalue\ to be considered}}.

The second filter requires check \pvalue\ fold change to be large enough. By default fold change threshold is equal to $5$. It means that only SNVs causing \pvalue\ change of 5x and more ($\mbox{FoldChange>5}$ or $\mbox{FoldChange<1/5}$) will be included in results. Fold change threshold can be specified using \cmdoption{--fold-change-cutoff}:\\
\cmdoption{--fold-change-cutoff \requiredarg{minimal fold change to be considered}}\\
or in short form:\\
\cmdoption{-F \requiredarg{minimal fold change to be considered}}

The last but the most useful option is \cmdoption{--precalc} which forces \cmdname{SNPScan} to work with precalculated \pvalue,thresholds pairs performing binary search to evaluate the \pvalue\ instead of calculating motif score distribution each time from scratch. It can reduce total computation time in hundreds of times for large datasets.
As an input it requires a folder with precalculated (\pvalue,threshold) pairs - one for each motif:\\
\cmdoption{--precalc \requiredarg{path to a folder with precalculated \pvalue, threshold pairs}}

These precalculated score distributions are to be produced by a \cmdname{PreprocessCollection} from APE toolbox.
Please refer to the respective section for details.


\example{}
\SNPScan{./hocomoco/pwms/ snp.txt --precalc ./collection\_thresholds}
\SNPScan{./hocomoco/pcms/ snp.txt --pcm --discretization 10 --background 0.2,0.3,0.3,0.2}

\subsubsection{Output data format}

\cmdname{SNPScan} prints all results to standard output, errors and messages go into standard error stream. First line of output is a header of table. Latter lines are rows of this table. Columns are:
\begin{itemize}
\item Name of sequence containing SNV
\item TF motif name
\item for the first allele variant:
\begin{itemize}
\item   the best position and strand of putative TF-DNA binding
\item   nucleotide word corresponding to the best binding sequence among all other words in sequence, intersecting SNV
\end{itemize}
\item the same two columns for the second allele variant
\item allele variants
\item \pvalue\ for the first allele variant
\item \pvalue\ for the second allele variant
\item fold change (the first \pvalue\ divided by the second \pvalue)
\end{itemize}

Position of the best binding place is given for the leftmost boundary of a binding sequence (independent of strand orientation). The SNV location is at zero, so the TFBS coordinates are always less or equal to zero. Strand is denoted as `direct` or `revcomp`. Words are given at the relevant strand (i.e. reverse-complement transformation is applied if necessary).
