\subsection{EvalSimilarity}
Stand-alone tool to compute the similarity for two given motifs with given thresholds or given \pvalue\ level. Uses \pvalue\ of 0.0005 by default i.e. compares sets of top 0.05\% of words ranked by PWM scores. 

In order to get feeling of Jaccard similarity scale and to better catch our output format, try these examples and take a look at corresponding logos:

\exampleof{rather similar motifs \texttt{KLF4\_f2} and \texttt{SP1\_f1}}
\EvalSimilarity{motifs/KLF4\_f2.pat\\ motifs/SP1\_f1.pat}
% \outputheader
% \cmdoutputfromfile{EvalSimilarityOutput(KLF4,SP1).txt}

\exampleof{the same motif \texttt{SP1\_f1} in opposite orientations}
\EvalSimilarity{motifs/SP1\_f1\_revcomp.pat\\ motifs/SP1\_f1.pat}
% \outputheader
% \cmdoutputfromfile{EvalSimilarityOutput(SP1,revcompSP1).txt}

\exampleof{significantly different motifs \texttt{SP1\_f1} and \texttt{GABPA\_f1}}
\EvalSimilarity{motifs/SP1\_f1.pat\\ motifs/GABPA\_f1.pat}
% \outputheader
% \cmdoutputfromfile{EvalSimilarityOutput(SP1,GABPA).txt}

Note! By default \cmdname{EvalSimilarity} selects the threshold corresponding to the \pvalue\ not 
less than requested (upper boundary). This is different from \cmdname{FindThreshold} which uses 
lower boundary for \pvalue. 

It is very important to select upper \pvalue\ boundary for short PWMs. In case of given 
low \pvalues\ they can recognize no words at all (so the Jaccard measure may have zero 
numerator and zero denominator). For reasonable threshold levels both upper and lower 
boundaries usually produce very close similarity values. 

Nevertheless, one can override this behavior with \cmdoption{‘--boundary lower’} option. In such a case if 
any of supplied PWMs recognizes no words for a selected \pvalue, then similarity can not be 
correctly determined and macroape will report the similarity value of $-1$. 

% The same note is also appliable to the search through a model collection (see below). 

% NOTE! For \cmdname{ScanCollection} tool this affects only a query PWM. The same option should be 
% specified with \cmdname{PrecalculateThresholds} when preprocessing a PWM collection

By default \cmdname{EvalSimilarity} tests all possible motif alignments in both orientations. But there is an option \cmdoption{--position} to evaluate similarity in an explicitly specified specified motif alignment. Option usage is following:

\cmdoption{--position \requiredarg{shift},\requiredarg{direct|revcomp}}

Option parameters are comma-separated, spaces not allowed.

Try the following examples:
\exampleof{rather similar motifs \texttt{KLF4\_f2} and \texttt{SP1\_f1} at optimal alignment}
\EvalSimilarity{motifs/KLF4\_f2.pat\\ motifs/SP1\_f1.pat\\ --position -1,direct}
% \outputheader
% \cmdoutputfromfile{EvalAlignmentOutput(RightPosition).txt}

\exampleof{rather similar motifs \texttt{KLF4\_f2} and \texttt{SP1\_f1} at completely wrong alignment}
\EvalSimilarity{motifs/KLF4\_f2.pat\\ motifs/SP1\_f1.pat\\ --position 3,revcomp}
% \outputheader
% \cmdoutputfromfile{EvalAlignmentOutput(WrongPosition).txt}
