\subsection{PrecalculateThresholds}
This tool is intended to process the motif collection (a folder containing separate
files for each motif) and to store precomputed score distributions of motif PWMs. Each score distributions is saved as a sorted list of \mbox{(threshold,\pvalue)} pairs with \pvalues\ taken at uniform intervals at quantiles of score distribution. It allows for faster score -- \pvalue\ conversion performing binary search through a list of thresholds or \pvalues. \cmdname{PrecalculateThresholds} doesn't store precise score distribution because for a non-disretized PWM it can be extremely large with unpractical precision. Practically it's sufficient to estimate \pvalue\ with a specified error level of e.g. 5\%.

In order to use precalculated distribution several *-APE tools have \cmdoption{--precalc} option which takes a folder containing results of \cmdname{PrecalculateThresholds}.

\textbf{Note:} Precalculation allows notably increase speed of threshold to \pvalue\ calculation (up to 100x). Unfortunately it deals with a file system to load the precalculated data. Thus it's recomended to use precalculated score distribution for tasks where the same motif \pvalue\ evaluation is performed multiple times so that the score distribution is loaded once and used multiple times. At a moment the only use case is ~-- \cmdname{perfectosape.SNPScan} which assesses each of multiple SNPs against the same motif collection.
% TODO: yet undocumented CollectDistanceMatrix also can take an advantage of PrecalculateThresholds (but there are more effective approaches).

\textbf{Note:} Resulting score distribution depends on a discretization level and on a specified background model. It is up to the user to control that a score distribution was precomputed with the proper parameters. The parameters values are not anyhow stored after score distribution precalculation and are not implicitly contolled when reusing precomputed data.

\usageheader
\PrecalculateThresholds{\requiredarg{motif collection folder}\\ \requiredarg{output folder}\\ \optionalarg{options}}

\example{}
\PrecalculateThresholds{./motifs/ ./motif\_thresholds/}

This will create \texttt{./motif\_thresholds/} folder (if not already exist) and multiple files inside, one file per motif in \texttt{./motifs/} folder. For a given motif output file will be named as \texttt{\inangles{name of motif}.thr}.
Each file contains lines in the following format:

\texttt{\inangles{threshold} \textit{tab} \inangles{corresponding \pvalue}}

Lines are sorted with thresholds ascending (\pvalue\ descending).

It takes about half a minute to preprocess the collection of $\sim$400 mononucleotide PWMs with default
parameters using 1.5 GHz CPU. During precalculation task progress will be printed to standard error stream. To suppress output use \cmdoption{--silent} option.

To alter granularity of resulting \pvalues\ list one can use \cmdoption{--pvalues} option in the following format:\\
\cmdoption{--pvalues \requiredarg{from,to,step,mode}}\\

Parameters set the \pvalues\ progression in the resulting list. \pvalues\ can use arithmetic or geometric progession which corresponds to \texttt{add} or \texttt{mul} value of \texttt{mode}.

\texttt{from} and \texttt{to} represent progression boundaries and \cmdoption{step} corresponds to a common difference (\texttt{add}) or a common ratio (\texttt{mul}) of progression. Parameters are comma-separated without spaces between.

For example, default progression can be written as follows:\\
\cmdoption{--pvalues 1.0,1e-15,1.05,mul}\\

It means that \cmdname{PrecalculateThresholds} collect thresholds for
each of these \pvalues: $1.0, 1.0/1.05, 1.0/1.05^2, 1.0/1.05^3,\ldots,
10^{-15}$

To specify relative error of $\epsilon$ use geometric progression with common ratio of $1+\epsilon$ and boundaries: from $1.0$ to a minimal expected non-zero \pvalue.
