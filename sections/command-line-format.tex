\subsection{Command line format}
All tools use similar command-line format. The examples are shown under the assumtion that the *-APE package \texttt{ape.jar} is located in the current folder (working directory). A typical command line will look like:\par
\texttt{\secondcolindent{java -cp ape.jar }{ru.autosome.ToolName\\ \requiredarg{required arguments}\\ \optionalarg{options}}}

Each tool can be used with \cmdoption{--help} or \cmdoption{-h} options to display a detailed help message describing order of arguments and a list of optional parameters.

Each tool is provided in mononucleotide and dinucleotide versions for mono- and diPWMs and respective background models.
Generally, mononucleotide version has wider application range, since most of existing motif collections provide only basic mononucleotide PWMs.
Naming convention is the same for all tools: mononucleotide version is located in package's root, dinucleotide version has the same name but is located in a subpackage \texttt{".di"}.

E.g. for \cmdname{ape.FindThreshold} the full class names are:
\begin{itemize}
\item\cmdname{ru.autosome.ape.FindThreshold} for mononucleotide version
\item\cmdname{ru.autosome.ape.di.FindThreshold} for dinucleotide version.
\end{itemize}

Please note, that dinucleotide tools use special input formats for dinucleotide Position Weight Matrices (diPWM) and respective background models. Input data formats are described in a special section.


\subsubsection{Output formats}

All tools except \cmdname{PrecalculateThresholds} print their results into the standard output stream (stdout).
\cmdname{PrecalculateThresholds} stores its results in a set of output files created in a specified folder.

For each tool the output can be redirected to a file using OS syntax, e.g. with a "\texttt{\textgreater}"-sign. For example:
\texttt{\FindPvalue{motifs/KLF4\_f2.pwm 3.3 5.0 7.1 \textgreater\ KLF4\_P-values.txt}}

Output generally consists of two types of lines. Lines starting with \texttt{"\#"} character (comments) show
input parameters and descriptions. The results are presented in non-commented lines.
