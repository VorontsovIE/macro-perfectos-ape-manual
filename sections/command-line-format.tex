\subsection{Command line format}
All tools follow common usage format which is to be described below. We suggest you have a package in a file named \texttt{ape.jar} and located in the current folder. Then typical call of a tool will look like:\par
\texttt{\secondcolindent{java -cp ape.jar }{ru.autosome.ToolName\\ \requiredarg{required arguments}\\ \optionalarg{options}}}

Each command being used with \cmdoption{--help} or \cmdoption{--help} option prints help string describing order of arguments and a list of optional arguments for each command.

Each tools is duplicated for working with mononucleotide and dinucleotide PWMs and corresponding background models. If not sure, use mononucleotide version. Naming convention is the same for all tools: mononucleotide version is located in package's root, dinucleotide version has the same name but is located in a subpackage \texttt{".di"}.

E.g. for \cmdname{ape.FindThreshold} tool, class names are:
\begin{itemize}
\item\cmdname{ru.autosome.ape.FindThreshold} for mononucleotide version
\item\cmdname{ru.autosome.ape.di.FindThreshold} for dinucleotide version.
\end{itemize}

Obviously, dinucleotide versions of tools use different input formats of Position Weight Matrices (DiPWM) and backgrounds. Input data formats will be discussed later.


\subsubsection{Output formats}

All tools except \cmdname{PrecalculateThresholds} print results into stdout (standard output stream).
\cmdname{PrecalculateThresholds} yields result to a set of output files in a specified folder. 

\cmdname{PrecalculateThresholds} and \cmdname{ScanCollection} print progress information to the stderr stream. This output can be disabled with \cmdoption{--silent} option.

Output generally consists of two line types. Lines starting with \texttt{"\#"} character (commented out lines) show 
input parameter values and descriptions. The results are presented in non-commented lines.
