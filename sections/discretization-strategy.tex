\section{PWM discretization}
Following the general idea described in [Touzet2007] we can effectively calculate the \pvalue\ 
for a given PWM having fixed precision and a given threshold value. The algorithm of Touzet 
efficiently processes matrices with integer elements, so the real value matrices are transformed 
into integer value matrices by multiplying on discretization constant and truncating the decimals. 
Effectively this is similar to rounding up real value PWMs leaving only the fixed number of 
decimal places. The higher discretization level will result in a more accurate \pvalue\ calculation 
and an increased computational time. 

Please note, that in contrast to the original Touzet algorithm here we applying "ceil" 
operation to the matrix elements (instead of "floor" in the original paper of Touzet). This allows 
us to have a strict upper boundary of the threshold for a given \pvalue.

We use the default discretization level of 10000 to perform calculation accuracy up to 
four significant digits for single-PWM tools. The straightforward discretization by rounding up to 
the nearest integer is used by default for a fast and rough search through the motif collection. 
The default level of 10 (one decimal place) is used for a more precise search of similar motifs. 

Thus in our case discretization is the transformation as follows: discretized~$S$ is
$S$~multiplied by discretization level $V$ and rounded up to the nearest integer value. 

\texttt{
\begin{tabular}{ll}
Example:\\
S = 1.6734\\
discretization V=1 & discretized S = ceil(1.6734) = 2\\
discretization V=10 & discretized S = ceil(16.734) = 17\\
discretization V=100 & discretized S = ceil(167.34) = 168
\end{tabular}
}

Discretization will generally preserve the word score ranking with the common exception 
for words that would obtain identical scores. The main advantage of the discretization 
is decreasing of the number of possible scores so the set of all possible scores can be 
enumerated more effectively.
