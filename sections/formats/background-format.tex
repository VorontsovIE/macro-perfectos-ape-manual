\subsection{Background model options}
Nucleotide frequencies of a background model can be specified in optional arguments, e.g. \cmdoption{-b} or \cmdoption{--query-background}. All background options use the same format with a single required argument: \cmdoption{-b \requiredarg{value}}.

Default background model is a \texttt{wordwise} model. It means that all our calculations assume uniform nucleotide distribution and the exact number of words is used everywhere instead of probabilities of a word set.
E.g. \texttt{FindPvalue} will calculate not the probability of a random word score to pass the threshold but a fraction of words scoring greater than threshold estimating the exact number of such words.

A number of words is a more natural and intuitive to use, especially if the background model cannot be properly selected
thus we suggest "wordwise" mode by default.

Wordwise mode can be specified explicitly, e.g. using \cmdoption{-b wordwise} key.

All following formats are different ways to specify frequencies of each nucleotide:
\begin{itemize}
\item The most simple nucleotide background model is uniform, each nucleotide has the same probability to occur. Option format is: \cmdoption{-b uniform}. This is close to wordwise mode, but word set probabilities are used and reported instead of raw counts of words.
\item It is also possible to specify a fixed GC-content~(in range 0 to 1): \cmdoption{-b \requiredarg{GC-content}}. E.g.~\cmdoption{"-b~0.6"}
\item The most detailed format is to explicitly specify nucleotide frequencies: \cmdoption{-b \requiredarg{$p_A,p_C,p_G,p_T$}}. E.g.~\cmdoption{"-b 0.2,0.3,0.3,0.2"} will define the same frequencies as for GC-content of 0.6. Note that nucleotide frequencies should be given in alphabetical ACGT-order separated with commas. 
\end{itemize}

\textbf{Note:} No spaces between frequencies are allowed (commas only). Sum of frequencies should be equal to 1.0.
