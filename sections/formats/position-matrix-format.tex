\subsection{Position matrix format description}
All tools described below use the following matrix file format (each binding site position 
corresponds to a separate line):

\texttt{\begin{tabular}{llll}
some\_header\\
pos1\_A\_weight & pos1\_C\_weight & pos1\_G\_weight & pos1\_T\_weight\\
\ldots\\
posw\_A\_weight & posw\_C\_weight & posw\_G\_weight & posw\_T\_weight
\end{tabular}
}

Position matrix format is appliable for all kinds of positional matrices: positional weight(PWM), count(PCM) and probability/frequency(PPM). Matrix elements meaning changes accordingly.

The number of the lines corresponds to the PWM width. If given, header will be treated as a motif name, otherwise filename will stand for motif name. Header has optional \texttt{"\textgreater"} sign at line start.

If necessary it's possible to read matrices, with nucleotides in rows and positions in columns. In order to do this, specify \cmdoption{--transpose} option.

\exampleof{PWM similar to HOCOMOCO transcription factor motif for KLF4}
{\small\verbatiminput{./MotifSamples/KLF4_f2_alike.pwm}}

More real-life examples are provided with the package.

Dinucleotide version of tools work with dinucleotide motifs. Dinucleotide positional matrices have similar format but contain 16 columns instead of 4. Columns go in order: AA, AC, AG, AT, CA, CC, \dots, TT. It's also possible to use mononucleotide motifs in dinucleotide tools. For rationales and details take a look at \cmdoption{--from-mono} option.
