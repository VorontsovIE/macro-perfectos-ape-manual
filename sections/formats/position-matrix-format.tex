\subsection{Position matrix format description}
All tools in the *-APE package use the following matrix file format (each binding site position 
corresponds to a separate line):

\texttt{\begin{tabular}{llll}
some\_header\\
pos1\_A\_weight & pos1\_C\_weight & pos1\_G\_weight & pos1\_T\_weight\\
\ldots\\
posw\_A\_weight & posw\_C\_weight & posw\_G\_weight & posw\_T\_weight
\end{tabular}
}

Position matrix format is appliable for all kinds of positional matrices: positional weight(PWM), count(PCM) and probability/frequency(PPM).
Positonal count matrices are allowed to contain floating point numbers (e.g. in the case the counts were derived from 
somehow weighted alignments).

The total number of lines corresponds to the PWM width (minus the header line). If given, header will be treated as a motif name, otherwise filename will stand for motif name. Header may carry an optional \texttt{"\textgreater"} sign at line start
(like in fasta files).

If necessary it's possible to read transposed matrices, with nucleotides in rows and positions in columns using \cmdoption{--transpose} option.

\exampleof{PWM similar to HOCOMOCO transcription factor motif for KLF4}
{\small\verbatiminput{./MotifSamples/KLF4_f2_alike.pwm}}

\exampleof{Transposed PWM similar to HOCOMOCO transcription factor motif for KLF4}
{\small\verbatiminput{./MotifSamples/KLF4_f2_alike_transposed.pwm}}

More real-life examples are provided with the package in respective motif collections.

Dinucleotide versions of *-APE tools use dinucleotide motifs. 
Dinucleotide positional matrices have similar format but contain 16 columns instead of 4. Columns go in order: AA, AC, AG, AT, CA, CC, \dots, TT. It's also possible to use mononucleotide motifs in dinucleotide tools (e.g. to use dinucleotide background). For rationales and details take a look at \cmdoption{--from-mono} option.
