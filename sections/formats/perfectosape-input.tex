\subsection{SNVs/SNPs format}
\cmdname{SNPScan} uses as input data a list of sequences containing SNVs.

List of sequences with SNVs is specified in a single file. Each sequence is situated on a separate line of the following format:

\texttt{\inangles{name of sequence with SNV} \inangles{sequence containing SNV}}

SNV name shouldn't contain spaces. Sequence consists of two allele variants in square brackets, separated with `/`, and flanking sequences at both sides. Length of flanking sequences should be sufficient to place the longest motif (so take 25-30bp at each side) into all positions relative to a place of polymorphism. 

\exampleof{SNV list}
\noindent\texttt{%
rs10040172 gatttgccctgattgcagttactga[G/A]tggtacagacatcgtaataatctta
rs10116271 taaattctatgtggggaagaggtct[C/T]gtagaggcgatgattcttacattgc
rs10208293 cttcatacatttatgtccagtacct[A/G]tggaccctccttgtgaactcttctc
rs10431961 tggcggggctggtcaggcggcgtcg[C/T]cggtacgctctgagcggcagcgtgt
}
