\section{Overview}
All tools are packed in a jar-file with compiled Java classes. There are three main packages for tools:
\texttt{ru.autosome.ape}, \texttt{ru.autosome.macroape} and \texttt{ru.autosome.perfectosape}.

\textbf{APE} in \texttt{ru.autosome.ape} stands for Approximate \pvalue\ Estimation, this package contains basic tools:
\begin{itemize}
\item \cmdname{FindThreshold}~--- to estimate a PWM score threshold for a given \pvalue
\item \cmdname{FindPvalue}~--- to estimate a PWM \pvalue\ corresponding to a given score threshold
\item \cmdname{PrecalculateThresholds}~--- to precalculate lists of thresholds tabulated by \pvalues\ for a given motif collection
\end{itemize}

\textbf{MACRO-APE} in \texttt{ru.autosome.macroape} denotes MAtrix CompaRisOn by Approximate \pvalue\ Estimation. Package consists of several tools related to motif comparison:
\begin{itemize}
\item \cmdname{EvalSimilarity}~--- to evaluate similarity for a given pair of PWMs.
\item \cmdname{ScanCollection}~--- to search a collection of motifs for PWMs similar to a given query.
% \item \cmdname{CollectDistanceMatrix}~--- to collect pairwise distance matrix for a motif collection which can be further used for motif clustering. (Not yet finalized tool!)
% \item \cmdname{AlignMotifs}~--- to evaluate the best motif relative alignment. (This tool not yet moved from ruby to Java!)
\end{itemize}

\textbf{PERFECTOS-APE} in \texttt{ru.autosome.perfectosape} denotes Predicting Regulatory Functional Effect of SNPs by Approximate \pvalue\ Estimation. Package contains a single tool:
\begin{itemize}
\item\cmdname{SNPScan}~--- to search a pack of sequences with SNVs or SNPs against a collection of PWMs for (SNV, PWM) pairs, such that single nucleotide substitution induces significant change of predicted affinity for a given PWM.
\end{itemize}

\textbf{Please note}, that *-APE tools by default consider all given matrices as positional weight matrices with additive scores already passed counts-to-weights transformation (e.g. log-odds). The usage of count matrices (PCMs) or frequency matrices (PPMs) is also possible with additional command-line keys (see the respective section).
